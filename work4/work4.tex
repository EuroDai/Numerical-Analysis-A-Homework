% Homework template for Inference and Information
% UPDATE: September 26, 2017 by Xiangxiang
\documentclass[a4paper]{article}
\usepackage[margin=1in]{geometry}
\usepackage{ctex}
\ctexset{
proofname = \heiti{证明}
}
\usepackage{amsmath, amssymb, amsthm}
% amsmath: equation*, amssymb: mathbb, amsthm: proof
\usepackage{moreenum}
\usepackage{mathtools}
\usepackage{url}
\usepackage{bm}
\usepackage{enumitem}
\usepackage{graphicx}
\usepackage{subcaption}
\usepackage{booktabs} % toprule
\usepackage[mathcal]{eucal}
\usepackage[thehwcnt = 4]{iidef} % 作业编号
\everymath{\displaystyle}


\thecourseinstitute{清华大学机械工程系}
\thecoursename{数值分析A}
\theterm{2025年秋季学期}
\hwname{作业}
\slname{\heiti{解}}
\begin{document}
\courseheader
\name{代卓远2025210205}

\begin{enumerate}
  \setlength{\itemsep}{4\parskip}

  % 题目1
  \item 下列向量序列$\{x^{(k)}\}$是否有极限?若有,写出其极限向量.
  \begin{enumerate}[label=(\arabic*)]
    \item $\boldsymbol{x}^{(k)}=\left(\mathrm{e}^{-k}\cos k,k\sin\frac{1}{k},3+\frac{1}{k^{2}}\right)^{\mathrm{T}}$
    \begin{solution}
      该向量序列有极限.\\
      $\lim\limits_{k\to\infty}\mathrm{e}^{-k}\cos k=0$,\\
      $\lim\limits_{k\to\infty}k\sin\frac{1}{k}=\lim\limits_{k\to\infty}\frac{\sin\frac{1}{k}}{\frac{1}{k}}=1$,\\
      $\lim\limits_{k\to\infty}3+\frac{1}{k^{2}}=3$.\\
      所以$\lim\limits_{k\to\infty}\boldsymbol{x}^{(k)}=\left(0,1,3\right)^{\mathrm{T}}$.
    \end{solution}
    \item $\boldsymbol{x}^{(k)}=\left(k\mathrm{e}^{-k^{2}},\frac{\cos k}{k},\sqrt{k^{2}+k}-k\right)^{\mathrm{T}}$
    \begin{solution}
      该向量序列有极限.\\
      $\lim\limits_{k\to\infty}k\mathrm{e}^{-k^{2}}=0$,\\
      $\lim\limits_{k\to\infty}\frac{\cos k}{k}=0$,\\
      $\lim\limits_{k\to\infty}\sqrt{k^{2}+k}-k=\lim\limits_{k\to\infty}\frac{(\sqrt{k^{2}+k}-k)(\sqrt{k^{2}+k}+k)}{\sqrt{k^{2}+k}+k}=\lim\limits_{k\to\infty}\frac{k}{\sqrt{k^{2}+k}+k}=\lim\limits_{k\to\infty}\frac{1}{\sqrt{1+\frac{1}{k}}+1}=\frac{1}{2}$.\\
      所以$\lim\limits_{k\to\infty}\boldsymbol{x}^{(k)}=\left(0,0,\frac{1}{2}\right)^{\mathrm{T}}$.
    \end{solution}
    
  \end{enumerate}

  % 题目8
  \item 分析方程组$$\begin{bmatrix}
  1 & a & 0 \\
  a & 1 & a \\
  0 & a & 1
  \end{bmatrix}
  \begin{bmatrix}
  x_1 \\
  x_2 \\
  x_3
  \end{bmatrix}
  =
  \begin{bmatrix}
  b_1 \\
  b_2 \\
  b_3
  \end{bmatrix}$$J法和GS法的收敛性.
  \begin{solution}
    \begin{align*}
      &a_{11}=1>0\\
      &\begin{vmatrix}1 & a \\ a & 1\end{vmatrix}=1-a^2>0\Rightarrow |a|<1\\
      &\begin{vmatrix}1 & a & 0 \\ a & 1 & a \\ 0 & a & 1\end{vmatrix}=1-2a^2>0\Rightarrow |a|<\frac{\sqrt{2}}{2}\\
    \end{align*}
    综上所述,当$|a|<\frac{\sqrt{2}}{2}$时收敛.\\
  \end{solution}

  % 题目12
  \item 设$A$对称正定 ,其特征值$\lambda_1\geq\lambda_2\geq\cdots\geq\lambda_n>0.$证明迭代法$$x^{(k+1)}\:=\:x^{(k)}\:+\:\omega(\:b\:-\:Ax^{(k)}\:)\:,\quad k\:=\:0\:,1\:,\cdots $$当 $\omega\in\left(0,\frac{2}{\lambda_{1}}\right)$时收敛,并讨论$\omega$取什么值时收敛最快?
  \begin{solution}
    迭代矩阵$B=I-\omega A$,其特征值为$\mu_i=1-\omega\lambda_i$,$i=1,2,\cdots,n$.\\
    收敛时,由充要条件$\rho(B)=\max\{|1-\omega\lambda_1|,|1-\omega\lambda_n|\}<1$.\\
    得到
    \[\begin{cases}
      |1-\omega\lambda_1|<1\\
      |1-\omega\lambda_n|<1
    \end{cases}\Rightarrow 0<\omega<\frac{2}{\lambda_1}.\]\\
    所以,当 $\omega\in\left(0,\frac{2}{\lambda_{1}}\right)$时收敛.\\
    由渐进收敛率$R(B)=-\ln\max\{|1-\omega\lambda_1|,|1-\omega\lambda_n|\}$
    \begin{align*}
      &|1-w\lambda_1|>|1-w\lambda_n|\\
      &\textcircled{1}1-w\lambda_n\leq0\Rightarrow\frac{1}{\lambda_n}\leq w<\frac{1}{\lambda_1}\\
      &\textcircled{2}
      \begin{cases}
        w\lambda_1-1\geqslant1-w\lambda_n\Rightarrow\frac{2}{\lambda_1+\lambda_n}\leqslant w\leqslant\frac{2}{\lambda_1}\\w\lambda_1-1\geqslant0\end{cases}\\
        &\left|1-w\lambda_n\right|>\left|1-w\lambda_1\right|\\
        &\textcircled{1}1-w\lambda_{1}\leq0\Rightarrow0<w\leq\frac{1}{\lambda_{1}}\\
        &\textcircled{2}\begin{cases}1-w\lambda_{n}\geq0\\1-w\lambda_{1}\leq0\end{cases}\Rightarrow\frac{1}{\lambda_{1}}\leq w\leq\frac{1}{\lambda_{n}}
    \end{align*}
  \end{solution}


  % 题目14
  \item 
  $$
  \left\{
  \begin{array}{rrrrr}
  10x_1 & - x_2 &              & = & 9,\\
  - x_1 & + 10x_2 & - 2x_3 & = & 7,\\
        & - 2x_2 & + 10x_3 & = & 6.
  \end{array}
  \right.
  $$
  \begin{enumerate}[label=(\arabic*)]
    \item 写出SOR法计算公式;
    \begin{solution}
      设$A=\begin{bmatrix}
      10 & -1 & 0 \\
      -1 & 10 & -2 \\
      0 & -2 & 10
      \end{bmatrix}$,$b=\begin{bmatrix}
      9 \\
      7 \\
      6
      \end{bmatrix}$,则$D=\begin{bmatrix}
      10 & 0 & 0 \\
      0 & 10 & 0 \\
      0 & 0 & 10
      \end{bmatrix}$,$L=\begin{bmatrix}
      0 & 0 & 0 \\
      -1 & 0 & 0 \\
      0 & -2 & 0
      \end{bmatrix}$,$U=\begin{bmatrix}
      0 & -1 & 0 \\
      0 & 0 & -2 \\
      0 & 0 & 0
      \end{bmatrix}$\\
      则SOR迭代格式为
      $$x^{(k+1)}=\left(D-\omega L\right)^{-1}\left[\omega b+\left(1-\omega\right)Dx^{(k)}+\omega Ux^{(k)}\right]$$
    \end{solution}
    \item 求最优松弛因子及$\omega=\omega_b$时SOR法的渐近收敛率;
    \begin{solution}
      $B_J=D^{-1}(L+U)=\begin{bmatrix}
      0 & -0.1 & 0 \\
      -0.1 & 0 & -0.2 \\
      0 & -0.2 & 0
      \end{bmatrix}$\\
      计算特征值,得$\lambda_1=\frac{\sqrt{5}}{10},\lambda_2=\frac{-\sqrt{5}}{10},\lambda_3=0$\\
      则$\rho(B_J)=\frac{\sqrt{5}}{10}$\\
      最优松弛因子$\omega_{b}=\frac{2}{1+\sqrt{1-\rho(B_J)^2}}=\frac{2}{1+\sqrt{1-\left(\frac{\sqrt{5}}{10}\right)^2}}=40-4\sqrt{95}$\\
      迭代矩阵$B=\left(D-\omega L\right)^{-1}\left[(1-\omega)D+\omega U\right]=\begin{bmatrix} - 0.0128& - 0.1013&0 \\ 0.0013& - 0.0026& - 0.2026 \\  - 0.0003&0.0005&0.0282\end{bmatrix}$\\
      谱半径$\rho(B)=0.0128$\\
      渐进收敛率$R(B)=-\ln\rho(B)=4.358$
    \end{solution}
    \item 取$x^{(0)}=\left(0,0,0\right)^{\mathrm{T}}$,用$\omega=\omega_b$的 SOR 法求 $x^{(1)},x^{(2)},x^{(3)}$
    \begin{solution}
      由(1)中的公式得\\$$x^{(1)} = \begin{bmatrix}0.911540\\ 0.801299\\ 0.770008\end{bmatrix}^T, 
x^{(2)} = \begin{bmatrix}0.981009\\ 0.954036\\ 0.791074\end{bmatrix}^T
x^{(3)} = \begin{bmatrix}0.995588\\ 0.957822\\ 0.791571\end{bmatrix}^T$$
    \end{solution}
  \end{enumerate}

  % 题目18(2)
  \item 取初始向量为零向量,用共轭梯度法解方程组
  $$\begin{bmatrix}4 & 3 & 0 \\ 3 & 4 & -1 \\ 0 & -1 & 4\end{bmatrix}\begin{bmatrix}x_{1} \\ x_{2} \\ x_{3}\end{bmatrix}=\begin{bmatrix}3 \\ 5 \\ -5\end{bmatrix}.$$
  \begin{solution}
    选取 $\begin{aligned}&x^{(0)}\in R^n,\\&p^{(0)}=r^{(0)}=b-Ax^{(0)},\end{aligned}$\\
    对 $k= 0, 1, \cdots$\\
   $\alpha_k=\displaystyle\frac{(r^{(k)},r^{(k)})}{(Ap^{(k)},p^{(k)})},$\\
   $x^{(k+1)}=x^{(k)}+\alpha_kp^{(k)},$\\
   $r^{(k+1)}=r^{(k)}-\alpha_kAp^{(k)},$\\
   $\beta_k=\displaystyle\frac{(r^{(k+1)},r^{(k+1)})}{(r^{(k)},r^{(k)})},$\\
   $p^{(k+1)}=r^{(k+1)}+\beta_kr^{(k)}.$\\
   第一次迭代:
\[
\alpha_0=\frac{r^{(0)T}r^{(0)}}{p^{(0)T}Ap^{(0)}}=\frac{59}{376}\approx0.1569.
\]
\[
x^{(1)}=x^{(0)}+\alpha_0 p^{(0)}=
\begin{bmatrix}
0.4707\\
0.7846\\
-0.7846
\end{bmatrix},\quad
r^{(1)}=r^{(0)}-\alpha_0 A p^{(0)}=
\begin{bmatrix}
-1.2367\\
-0.3351\\
-1.0771
\end{bmatrix}.
\]

第二次迭代:
\[
\alpha_1=\frac{r^{(1)T}r^{(1)}}{p^{(1)T}A p^{(1)}}\approx0.2311.
\]
\[
x^{(2)}=x^{(1)}+\alpha_1 p^{(1)}=
\begin{bmatrix}
0.2179\\
0.7620\\
-1.0884
\end{bmatrix},\quad
r^{(2)}=
\begin{bmatrix}
-0.1575\\
0.2100\\
0.1155
\end{bmatrix}.
\]

第三次迭代:
\[
\alpha_2\approx1.1490.
\]
\[
x^{(3)}=x^{(2)}+\alpha_2 p^{(2)}=
\begin{bmatrix}
0\\
1\\
-1
\end{bmatrix},\quad
r^{(3)}=0.
\]
  \end{solution}
\end{enumerate}
\end{document}

%%% Local Variables:
%%% mode: late\rvx
%%% TeX-master: t
%%% End:
