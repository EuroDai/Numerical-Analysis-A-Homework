% Homework template for Inference and Information
% UPDATE: September 26, 2017 by Xiangxiang
\documentclass[a4paper]{article}
\usepackage[margin=1in]{geometry}
\usepackage{ctex}
\ctexset{
proofname = \heiti{证明}
}
\usepackage{amsmath, amssymb, amsthm}
\usepackage{mathrsfs}
% amsmath: equation*, amssymb: mathbb, amsthm: proof
\usepackage{extarrows}
\usepackage{moreenum}
\usepackage{mathtools}
\usepackage{url}
\usepackage{bm}
\usepackage{enumitem}
\usepackage{graphicx}
\usepackage{subcaption}
\usepackage{booktabs} % toprule
\usepackage[mathcal]{eucal}
\usepackage[thehwcnt = 14]{iidef} % 作业编号
\everymath{\displaystyle}


\thecourseinstitute{清华大学机械工程系}
\thecoursename{数值分析A}
\theterm{2025年秋季学期}
\hwname{作业}
\slname{\heiti{解}}
\begin{document}
\courseheader
\name{代卓远2025210205}

\begin{enumerate}
  \setlength{\itemsep}{4\parskip}

  % 题目12
  \item 试推导Hamming公式$$y_{n+3} = \frac{1}{8}(9y_{n+2} - y_{n}) + \frac{3}{8}h[f(x_{n+3}, y_{n+3}) + 2f(x_{n+2}, y_{n+2}) - f(x_{n+1}, y_{n+1})]$$的局部截断误差主项.
  \begin{solution}

  \end{solution}

  % 题目13
  \item 试用数值积分方法直接推导二步方法$$y_{n+2}-y_{n+1}=\frac{h}{12}[5f(x_{n+2},y_{n+2})+8f(x_{n+1},y_{n+1})-f(x_{n},y_{n})].$$
  \begin{solution}

  \end{solution}



  % 题目15
  \item 证明线性二步法$$y_{n+2}+(b-1)y_{n+1}-by_{n}=\frac{1}{4}h[(b+3)f(x_{n+2},y_{n+2})+(3b+1)f(x_{n},y_{n})]$$当$b\neq-1$时是二阶的 ,当$b=-1$时是三阶的
  \begin{solution}
    
  \end{solution}



  % 题目17
  \item 讨论线性多步法$$y_{n+3}+\frac{1}{4}y_{n+2}-\frac{1}{2}y_{n+1}-\frac{3}{4}y_{n}=\frac{1}{8}h[19f(x_{n+2},y_{n+2})+5f(x_{n},y_{n})]$$的收敛性。
  \begin{solution}

  \end{solution}

  % 题目20
  \item 试证明隐式 Euler 方法是 A-稳定的.
  \begin{solution}

  \end{solution}

\end{enumerate}
\end{document}

%%% Local Variables:
%%% mode: late\rvx
%%% TeX-master: t
%%% End:
