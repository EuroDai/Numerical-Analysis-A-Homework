% Homework template for Inference and Information
% UPDATE: September 26, 2017 by Xiangxiang
\documentclass[a4paper]{article}
\usepackage[margin=1in]{geometry}
\usepackage{ctex}
\ctexset{
proofname = \heiti{证明}
}
\usepackage{amsmath, amssymb, amsthm}
\usepackage{mathrsfs}
% amsmath: equation*, amssymb: mathbb, amsthm: proof
\usepackage{extarrows}
\usepackage{moreenum}
\usepackage{mathtools}
\usepackage{url}
\usepackage{bm}
\usepackage{enumitem}
\usepackage{graphicx}
\usepackage{subcaption}
\usepackage{booktabs} % toprule
\usepackage[mathcal]{eucal}
\usepackage[thehwcnt = 9]{iidef} % 作业编号
\everymath{\displaystyle}


\thecourseinstitute{清华大学机械工程系}
\thecoursename{数值分析A}
\theterm{2025年秋季学期}
\hwname{作业}
\slname{\heiti{解}}
\begin{document}
\courseheader
\name{代卓远2025210205}

\begin{enumerate}
  \setlength{\itemsep}{4\parskip}

  % 题目1
  \item 设$[a,b] = [-1,1]$,$\rho(x) \equiv 1$,$P_0(x) \equiv 1$,试用Gram-Schmidt方法计算$P_1$,$P_2$,$P_3$
  \begin{solution}
    \[
      P_1(x) = x - \frac{(x, P_0)}{(P_0, P_0)} P_0 = x - \frac{\int_{-1}^1 x \cdot 1 \, dx}{\int_{-1}^1 1^2 \, dx} \cdot 1 = x - 0 = x
    \]
    \begin{align*}
      P_2(x) = x^2 - \frac{(x^2, P_0)}{(P_0, P_0)} P_0 - \frac{(x^2, P_1)}{(P_1, P_1)} P_1&= x^2 - \frac{\int_{-1}^1 x^2 \cdot 1 \, dx}{\int_{-1}^1 1^2 \, dx} \cdot 1 - \frac{\int_{-1}^1 x^2 \cdot x \, dx}{\int_{-1}^1 x^2 \, dx} \cdot x\\
      &= x^2 - \frac{\frac{2}{3}}{2} - 0 = x^2 - \frac{1}{3}
    \end{align*}
    \begin{align*}
      P_3(x) &= x^3 - \frac{(x^3, P_0)}{(P_0, P_0)} P_0 - \frac{(x^3, P_1)}{(P_1, P_1)} P_1 - \frac{(x^3, P_2)}{(P_2, P_2)} P_2\\
      &= x^3 - \frac{\int_{-1}^1 x^3 \cdot 1 \, dx}{\int_{-1}^1 1^2 \, dx} \cdot 1 - \frac{\int_{-1}^1 x^3 \cdot x \, dx}{\int_{-1}^1 x^2 \, dx} \cdot x - \frac{\int_{-1}^1 x^3 (x^2 - \frac{1}{3}) \, dx}{\int_{-1}^1 (x^2 - \frac{1}{3})^2 \, dx} (x^2 - \frac{1}{3})\\
      &= x^3 - 0 - \frac{\frac{2}{5}}{\frac{2}{3}} x - 0 = x^3 - \frac{3}{5} x
    \end{align*}
  \end{solution}

  % 题目2
  \item 设区间$[0,+\infty)$上权函数$\rho(x)=\mathrm{e}^{-x}$, 其正交多项式为Laguerre多项式,已知$L_0(x)\equiv1$, 试用Gram-Schmidt方法计算$L_1$, 并用递推关系式计算出$L_2$及$L_3.$
  \begin{solution}
    \[
      L_1(x) = x - \frac{(x, L_0)}{(L_0, L_0)} L_0 = x - \frac{\int_0^{+\infty} x e^{-x} \, dx}{\int_0^{+\infty} e^{-x} \, dx} \cdot 1 = x - 1
    \]
    递推关系式为
    \[
      L_{n+1}(x)=(1+2n-x)L_n(x)-n^2L_{n-1}(x)
    \]
    则
    \[
      L_2(x) = (1 + 2 - x)L_1(x) - 1^2 L_0(x) = (3 - x)(x - 1) - 1 = -x^2 + 4x - 4
    \] 
    \[
      L_3(x) = (1 + 4 - x)L_2(x) - 2^2 L_1(x) = (5 - x)(-x^2 + 4x - 4) - 4(x - 1) = x^3 - 9x^2 + 20x - 16
    \]
  \end{solution}

  % 题目5
  \item 用Chebyshev多项式$T_3$的零点在[-1,1]上对函数$f(x)=\ln(2+x)$构造二次Lagrange插值项式并估计其误差界.
  \begin{solution}
    $T_3$的零点为
    \[
      x_k = \cos\frac{(2k-1)\pi}{6} \pi, \quad k=1,2,3
    \]
    即
    \[
      x_1 = \frac{\sqrt{3}}{2}, \quad x_2 = 0, \quad x_3 = -\frac{\sqrt{3}}{2}
    \]
    则$f(x_1)=\ln(2+\frac{\sqrt{3}}{2})=1.05293$, $f(x_2)=\ln 2=0.69315$, $f(x_3)=\ln(2-\frac{\sqrt{3}}{2})=0.12573$,插值多项式为
    \begin{align*}
      L_2(x) &= f(x_1) \frac{(x - x_2)(x - x_3)}{(x_1 - x_2)(x_1 - x_3)} + f(x_2) \frac{(x - x_1)(x - x_3)}{(x_2 - x_1)(x_2 - x_3)} + f(x_3) \frac{(x - x_1)(x - x_2)}{(x_3 - x_1)(x_3 - x_2)} \\
      &= 1.05293 \frac{(x - 0)(x + \frac{\sqrt{3}}{2})}{(\frac{\sqrt{3}}{2} - 0)(\frac{\sqrt{3}}{2} + \frac{\sqrt{3}}{2})} + 0.69315 \frac{(x - \frac{\sqrt{3}}{2})(x + \frac{\sqrt{3}}{2})}{(0 - \frac{\sqrt{3}}{2})(0 + \frac{\sqrt{3}}{2})} + 0.12573 \frac{(x - \frac{\sqrt{3}}{2})(x - 0)}{(-\frac{\sqrt{3}}{2} - \frac{\sqrt{3}}{2})(-\frac{\sqrt{3}}{2} - 0)} \\
      &\approx 0.69315+0.53532x-0.13843x^2
      \end{align*}
      误差界为
      \[
        \|f-L_n\|_{\infty}\leqslant\frac{h^{n+1}}{4(n+1)}\|f^{(n+1)}\|_{\infty}=\frac{h^{3}}{12}\|f^{(3)}\|_{\infty}\approx 0.1083
      \]
  \end{solution}

  % 题目6
  \item 设$T_{n}$是 $n$ 次 Chebyshev 多项式, 令 $T_{n}^{*}(x)=T_{n}(2x-1)$, $x\in [0,1]$, 试证明 $\{T_{n}^{*}\}$是[0, 1]上权函数$\rho(x)=\frac{1}{\sqrt{x-x^{2}}}$的正交多项式.
  \begin{solution}
    \[
      T_{n}^{*}(x) = T_n(2x - 1)= \cos(n \arccos(2x - 1))
    \]
    则
    \begin{align*}
      \int_0^1 T_n^*(x) T_m^*(x) \rho(x) \, dx &= \int_0^1 \cos(n \arccos(2x - 1)) \cos(m \arccos(2x - 1)) \frac{1}{\sqrt{x - x^2}} \, dx \\
      &\xlongequal{t = 2x - 1} \frac{1}{2} \int_{-1}^1 \cos(n \arccos t) \cos(m \arccos t) \frac{1}{\sqrt{\frac{t + 1}{2} - (\frac{t + 1}{2})^2}} \, dt \\
      &= \int_{-1}^1 \cos(n \arccos t) \cos(m \arccos t) \frac{1}{\sqrt{1 - t^2}} \, dt
    \end{align*}
    由于Chebyshev多项式$T_n$在$[-1,1]$上关于权函数$\rho(t) = \frac{1}{\sqrt{1 - t^2}}$正交,因此$\{T_n^*\}$在$[0,1]$上关于权函数$\rho(x) = \frac{1}{\sqrt{x - x^2}}$正交。
  \end{solution}

  % 题目8
  \item 设$f(x) = x\ln x$, $x \in [1, 3]$, 试求出$f$在$\mathscr{P}_1$和$\mathscr{P}_2$中的最佳平方逼近多项式$P_1^*$和$P_2^*$。
  \begin{solution}
    在$\mathscr{P}_1$中,设$P_1^*(x) = a_0 + a_1 x$,则
    \[
      \begin{cases}
        (f - P_1^*, 1) = 0 \\
        (f - P_1^*, x) = 0
      \end{cases}
      \Rightarrow
      \begin{cases}
        \int_1^3 (x \ln x - a_0 - a_1 x) \, dx = 0 \\
        \int_1^3 (x \ln x - a_0 - a_1 x) x \, dx = 0
      \end{cases}
    \]
    解得$a_0 = \frac{9}{4}\ln 3 - \frac{13}{3}$,$a_1 = \frac{5}{3}.$,则
    \[
      P_1^*(x) = \left(\frac{9}{4}\ln 3 - \frac{13}{3}\right) + \frac{5}{3} x
    \]
    在$\mathscr{P}_2$中,设$P_2^*(x) = b_0 + b_1 x + b_2 x^2$,则
    \[
      \begin{cases}
        (f - P_2^*, 1) = 0 \\
        (f - P_2^*, x) = 0 \\
        (f - P_2^*, x^2) = 0
      \end{cases}
      \Rightarrow
      \begin{cases}
        \int_1^3 (x \ln x - b_0 - b_1 x - b_2 x^2) \, dx = 0 \\
        \int_1^3 (x \ln x - b_0 - b_1 x - b_2 x^2) x \, dx = 0 \\
        \int_1^3 (x \ln x - b_0 - b_1 x - b_2 x^2) x^2 \, dx = 0
      \end{cases}
    \]
    解得$\begin{cases}
b_0 &= \frac{567}{32}\ln 3 - \frac{163}{8},\\
b_1 &= \frac{115}{6} - \frac{135}{8}\ln 3,\\
b_2 &= \frac{135}{32}\ln 3 - \frac{35}{8}.
\end{cases}$,则
    \[
      P_2^*(x) = \left(\frac{567}{32}\ln 3 - \frac{163}{8}\right) + \left(\frac{115}{6} - \frac{135}{8}\ln 3\right) x + \left(\frac{135}{32}\ln 3 - \frac{35}{8}\right) x^2
    \]
  \end{solution}
\end{enumerate}
\end{document}

%%% Local Variables:
%%% mode: late\rvx
%%% TeX-master: t
%%% End:
