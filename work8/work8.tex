% Homework template for Inference and Information
% UPDATE: September 26, 2017 by Xiangxiang
\documentclass[a4paper]{article}
\usepackage[margin=1in]{geometry}
\usepackage{ctex}
\ctexset{
proofname = \heiti{证明}
}
\usepackage{amsmath, amssymb, amsthm}
% amsmath: equation*, amssymb: mathbb, amsthm: proof
\usepackage{extarrows}
\usepackage{moreenum}
\usepackage{mathtools}
\usepackage{url}
\usepackage{bm}
\usepackage{enumitem}
\usepackage{graphicx}
\usepackage{subcaption}
\usepackage{booktabs} % toprule
\usepackage[mathcal]{eucal}
\usepackage[thehwcnt = 8]{iidef} % 作业编号
\everymath{\displaystyle}


\thecourseinstitute{清华大学机械工程系}
\thecoursename{数值分析A}
\theterm{2025年秋季学期}
\hwname{作业}
\slname{\heiti{解}}
\begin{document}
\courseheader
\name{代卓远2025210205}

\begin{enumerate}
  \setlength{\itemsep}{4\parskip}

  % 题目10
  \item 设$f(x)=e^{0.1x^2}$,用基函数的Lagrange形式的三次Hermite插值多项式$H_3$来计算$f(1.25)$的近似值. $f$及$f'$的值如下
  $$\begin{array}{c|cc}
    \hline
    x_i & 1 & 1.5 \\
    \hline
    f(x_i) & 1.105170918 & 1.252322716 \\
    f'(x_i) & 0.2210341836 & 0.3756968148 \\
    \hline
  \end{array}$$
  \begin{solution}
    \begin{align*}H_{3}(x)&=\sum_{j=0}^1[f(x_j)\alpha_j(x)+f_j'(x)\beta_j(x)]\\&=f(x_0)\alpha_0(x)+f'(x_0)\beta_0(x)+f(x_1)\alpha_1(x)+f'(x_1)\beta_1(x)\end{align*}
    其中
    \begin{align*}
      \alpha_0(x)&=\left(1-2(x-x_0)\sum_{i=1}^1\frac{1}{x_0-x_i}\right)l_0^2(x)=(4x-3)l_0^2(x)\\
      \beta_0(x)&=(x-x_0)l_0^2(x)=(x-1)l_0^2(x)\\
      \alpha_1(x)&=\left(1-2(x-x_1)\sum_{i=0,i\neq 1}^1\frac{1}{x_1-x_i}\right)l_1^2(x)=(7-4x)l_1^2(x)\\
      \beta_1(x)&=(x-x_1)l_1^2(x)=(x-1.5)l_1^2(x)
    \end{align*}
    基函数
    \begin{align*}
      l_0(x)&=\frac{x-x_1}{x_0-x_1}=\frac{x-1.5}{1-1.5}=-2(x-1.5)\\
      l_1(x)&=\frac{x-x_0}{x_1-x_0}=\frac{x-1}{1.5-1}=2(x-1)
    \end{align*}
  \end{solution}
  得到
  \begin{align*}
    H_3(x)&=1.105170918(4x-3)(-2(x-1.5))^2+0.2210341836(x-1)(-2(x-1.5))^2\\
    &+1.252322716(7-4x)(2(x-1))^2+0.3756968148(x-1.5)(2(x-1))^2\\
    &\approx 0.0324952256x^3+0.0328055352x^2+0.0579374364x+0.9819327208
  \end{align*}
  则
  \[f(1.25)\approx H_3(1.25)\approx 1.16908040255\]

  % 题目11
  \item 设$f(x)=\sin(e^x-2)$, 其数据如下$$\begin{array}{c|cc}
    \hline
    x & f(x) & f'(x) \\
    \hline
    0.8 & 0.22363362 & 2.1691753 \\
    1.0 & 0.65809197 & 2.0466965 \\
    \hline
  \end{array}$$利用Newton形式的三次Hermite插值多项式$H_3$来计算$f(0.9)$的近似值, 并计算其实际误差及相应的误差估计.
  \begin{solution}
    均差
    \begin{align*}
      f[x_0]&=f(0.8)=0.22363362\\
      f[x_0,x_0]&=f'(0.8)=2.16917530\\
      f[x_0,x_0,x_1]&=\frac{f[x_0,x_1]-f[x_0,x_0]}{x_1-x_0}=\frac{\frac{f(1.0)-f(0.8)}{1.0-0.8}-2.1691753}{1.0-0.8}=0.01558225\\
      f[x_0,x_0,x_1,x_1]&=\frac{f[x_0,x_1,x_1]-f[x_0,x_0,x_1]}{x_1-x_0}=\frac{\frac{f'(1.0)-f[x_0,x_1]}{1.0-0.8}+5.05541865}{1.0-0.8}=−3.21779250
    \end{align*}
    则Newton形式的三次Hermite插值多项式为
    \begin{align*}
      H_3(x)&=f(x_0)+f[x_0,x_0](x-x_0)+f[x_0,x_0,x_1](x-x_0)^2+f[x_0,x_0,x_1,x_1](x-x_0)^2(x-x_1)\\
      &=0.22363362+2.16917530(x-0.8)+0.01558225(x-0.8)^2−3.21779250(x-0.8)^2\\
      &\approx -3.21779250x^3+8.38184275x^2-5.06361150x+0.55765322
    \end{align*}
    则
    \[f(0.9)\approx H_3(0.9)\approx 0.44392476\]
    实际误差
    \[f(0.9)-H_3(0.9)\approx -0.0003\]
    误差估计
    \[R_3(0.9)=f[x_0,x_0,x_1,x_1,0.9](x-0.8)^2(0.9−1.0)^2=−0.0003323\]
  \end{solution}

  % 题目13
  \item 求次数不超过4次的多项式P, 使其满足$$P(1)=P(3)=0, \quad P(2)=1, \quad P'(1)=0, \quad P''(1)=8;$$并写出其Newton形式的余项.
  \begin{solution}
    设 Hermite 插值节点为
      $x_0=1,\;x_1=1,\;x_2=1,\;x_3=2,\;x_4=3,$
    定义 $f(x)=P(x)$,则
    \[
      f(1)=0,\quad f'(1)=0,\quad f''(1)=8,\quad f(2)=1,\quad f(3)=0.
    \]
    均差
    \[
      f[x_0]=f(1)=0,
    \]
    \[
      f[x_0,x_1]=f'(1)=0 \quad (x_0=x_1=1),
    \]
    \[
      f[x_0,x_1,x_2]=\frac{f''(1)}{2!}=\frac{8}{2}=4 \quad (x_0=x_1=x_2=1).
    \]
    \[
      f[x_2,x_3]=\frac{f(2)-f(1)}{2-1}
      =\frac{1-0}{1}=1,
    \]
    \[
      f[x_1,x_2]=f'(1)=0,
    \]
    \[
      f[x_1,x_2,x_3]
      =\frac{f[x_2,x_3]-f[x_1,x_2]}{x_3-x_1}
      =\frac{1-0}{2-1}=1,
    \]
    \[
      f[x_0,x_1,x_2,x_3]
      =\frac{f[x_1,x_2,x_3]-f[x_0,x_1,x_2]}{x_3-x_0}
      =\frac{1-4}{2-1}=-3.
    \]
    \[
      f[x_2,x_3]=1,\quad
      f[x_3,x_4]=\frac{f(3)-f(2)}{3-2}=\frac{0-1}{1}=-1,
    \]
    \[
      f[x_2,x_3,x_4]
      =\frac{f[x_3,x_4]-f[x_2,x_3]}{x_4-x_2}
      =\frac{-1-1}{3-1}=-1,
    \]
    \[
      f[x_1,x_2,x_3,x_4]
      =\frac{f[x_2,x_3,x_4]-f[x_1,x_2,x_3]}{x_4-x_1}
      =\frac{-1-1}{3-1}=-1,
    \]
    \[
      f[x_0,x_1,x_2,x_3,x_4]
      =\frac{f[x_1,x_2,x_3,x_4]-f[x_0,x_1,x_2,x_3]}{x_4-x_0}
      =\frac{-1-(-3)}{3-1}=1.
    \]

    因此 Newton 形式为
    \begin{align*}
      P(x)&=f[x_0]
        +f[x_0,x_1](x-x_0)
        +f[x_0,x_1,x_2](x-x_0)(x-x_1)\\
        &\quad
        +f[x_0,x_1,x_2,x_3](x-x_0)(x-x_1)(x-x_2)\\
        &\quad
        +f[x_0,x_1,x_2,x_3,x_4](x-x_0)(x-x_1)(x-x_2)(x-x_3)\\
        &=0+0(x-1)+4(x-1)^2-3(x-1)^3+(x-1)^3(x-2).
        &=x^4-8x^3+22x^2-24x+9.
    \end{align*}
    余项为
    \[
      R_4(x)
      =f[1,1,1,2,3,x]\,(x-1)^3(x-2)(x-3).
    \]
  \end{solution}

  % 题目14
  \item 设$f(x)=\frac{1}{a-x}$, 证明
  \begin{enumerate}[label=(\arabic*)]
    \item $f[x_0,x_1,\cdots,x_n]=\prod_{j=0}^{n}\left(\frac{1}{a-x_j}\right)$.
    \begin{proof}
      由数学归纳法\\
      当$n=0$时, $f[x_0]=f(x_0)=\frac{1}{a-x_0}$, 命题成立.\\
      假设当$n=k$时, 命题成立, 即$f[x_0,x_1,\cdots,x_k]=\prod_{j=0}^{k}\left(\frac{1}{a-x_j}\right)$.\\
      则当$n=k+1$时,\\
      \begin{align*}
        f[x_0,x_1,\cdots,x_{k+1}]&=\frac{f[x_1,x_2,\cdots,x_{k+1}]-f[x_0,x_1,\cdots,x_k]}{x_{k+1}-x_0}\\
        &=\frac{\prod_{j=1}^{k+1}\left(\frac{1}{a-x_j}\right)-\prod_{j=0}^{k}\left(\frac{1}{a-x_j}\right)}{x_{k+1}-x_0}\\
        &=\frac{\frac{(a-x_0)-(a-x_{k+1})}{(a-x_0)(a-x_{k+1})}\prod_{j=1}^{k}\left(\frac{1}{a-x_j}\right)}{x_{k+1}-x_0}\\
        &=\frac{1}{(a-x_0)(a-x_{k+1})}\prod_{j=1}^{k}\left(\frac{1}{a-x_j}\right)\\
        &=\prod_{j=0}^{k+1}\left(\frac{1}{a-x_j}\right)
      \end{align*}
      故命题成立.
    \end{proof}
    \item $\frac{1}{a-x}=\frac{1}{a-x_0}+\frac{1}{(a-x_0)(a-x_1)}(x-x_0)+\cdots+\frac{1}{(a-x_0)\cdots(a-x_n)}(x-x_0)\cdots$
    $$(x-x_{n-1})+\frac{1}{(a-x_0)\cdots(a-x_n)(a-x)}(x-x_0)\cdots(x-x_n)$$
    \begin{proof}
      由Newton插值多项式的定义可知
      \begin{align*}
        P_n(x)&=f[x_0]+f[x_0,x_1](x-x_0)+\cdots+f[x_0,x_1,\cdots,x_n](x-x_0)\cdots(x-x_{n-1})\\
        &=\frac{1}{a-x_0}+\frac{1}{(a-x_0)(a-x_1)}(x-x_0)+\cdots+\frac{1}{(a-x_0)\cdots(a-x_n)}(x-x_0)\cdots(x-x_{n-1})
      \end{align*}
      由插值余项公式可知
      \[R_n(x)=f[x_0,x_1,\cdots,x_n,x](x-x_0)(x-x_1)\cdots(x-x_n)=\frac{1}{(a-x_0)\cdots(a-x_n)(a-x)}(x-x_0)\cdots(x-x_n)\]
      故
      \begin{align*}
        f(x)&=P_n(x)+R_n(x)\\
        &=\frac{1}{a-x_0}+\frac{1}{(a-x_0)(a-x_1)}(x-x_0)+\cdots+\frac{1}{(a-x_0)\cdots(a-x_n)}(x-x_0)\cdots(x-x_{n-1})\\
        &+\frac{1}{(a-x_0)\cdots(a-x_n)(a-x)}(x-x_0)\cdots(x-x_n)
      \end{align*}
    \end{proof}
  \end{enumerate}

  % 题目19
  \item 给定数据表$$\begin{array}{c|cccc}
    \hline
    x & -3 & -2 & 1 & 4 \\
    \hline
    f(x) & 2 & 0 & 3 & 1 \\
    \hline
  \end{array}$$
  试求三次样条插值函数 S, 使其分别满足边界条件
  \begin{enumerate}[label=(\arabic*)]
    \item $S'(x_0)=-1, S'(x_3)=1.$
    \begin{solution}
      \begin{align*}
        S_0(x)&=a_0+b_0(x+3)+c_0(x+3)^2+d_0(x+3)^3\\
        S_1(x)&=a_1+b_1(x+2)+c_1(x+2)^2+d_1(x+2)^3\\
        S_2(x)&=a_2+b_2(x-1)+c_2(x-1)^2+d_2(x-1)^3
      \end{align*}
      由插值条件得
      \begin{align*}
        S_0(-3)&=a_0=2\\
        S_0(-2)&=a_0+b_0+c_0+d_0=0\\
        S_1(-2)&=a_1=0\\
        S_1(1)&=a_1+3b_1+9c_1+27d_1=3\\
        S_2(1)&=a_2=3\\
        S_2(4)&=a_2+3b_2+9c_2+27d_2=1
      \end{align*}
      由连续性条件得
      \begin{align*}
        S_0'(-2)&=b_0+2c_0+3d_0=S_1'(-2)=b_1\\
        S_1'(1)&=b_1+6c_1+27d_1=S_2'(1)=b_2
      \end{align*}
      由二阶连续性条件得
      \begin{align*}
        S_0''(-2)&=2c_0+6d_0=S_1''(-2)=2c_1\\
        S_1''(1)&=2c_1+18d_1=S_2''(1)=2c_2\\
        S_2'(4)=b_2+6c_2+27d_2=1.
      \end{align*}
      由边界条件得
      \begin{align*}
        S_0'(-3)&=b_0=-1\\
        S_2'(4)&=b_2+6c_2+27d_2=1
      \end{align*}
      解得
      \begin{align*}
        S_0(x)&=2- (x+3)-\frac{76}{31}(x+3)^2+\frac{45}{31}(x+3)^3\\
        S_1(x)&=-\frac{48}{31}(x+2)+\frac{59}{31}(x+2)^2-\frac{98}{279}(x+2)^3\\
        S_2(x)&=3+\frac{12}{31}(x-1)-\frac{39}{31}(x-1)^2+\frac{253}{837}(x-1)^3.
      \end{align*}
    \end{solution}
    \item $S''(x_0)=0, S''(x_3)=0.$
    \begin{solution}
      同理可得
      \begin{align*}
        S_0(x)&=2-\frac{215}{87}(x+3)+\frac{41}{87}(x+3)^3,\quad &-3\le x\le -2\\
        S_1(x)&=-\frac{92}{87}(x+2)+\frac{41}{29}(x+2)^2-\frac{190}{783}(x+2)^3,\quad &-2\le x\le 1\\
        S_2(x)&=3+\frac{76}{87}(x-1)-\frac{67}{87}(x-1)^2+\frac{67}{783}(x-1)^3,\quad &1\le x\le 4.
      \end{align*}
    \end{solution}
  \end{enumerate}
\end{enumerate}
\end{document}

%%% Local Variables:
%%% mode: late\rvx
%%% TeX-master: t
%%% End:
