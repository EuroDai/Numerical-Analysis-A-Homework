% Homework template for Inference and Information
% UPDATE: September 26, 2017 by Xiangxiang
\documentclass[a4paper]{article}
\usepackage[margin=1in]{geometry}
\usepackage{ctex}
\ctexset{
proofname = \heiti{证明}
}
\usepackage{amsmath, amssymb, amsthm}
% amsmath: equation*, amssymb: mathbb, amsthm: proof
\usepackage{moreenum}
\usepackage{mathtools}
\usepackage{url}
\usepackage{bm}
\usepackage{enumitem}
\usepackage{graphicx}
\usepackage{subcaption}
\usepackage{booktabs} % toprule
\usepackage[mathcal]{eucal}
\usepackage[thehwcnt = 2]{iidef} % 作业编号
\everymath{\displaystyle}


\thecourseinstitute{清华大学机械工程系}
\thecoursename{数值分析A}
\theterm{2025年秋季学期}
\hwname{作业}
\slname{\heiti{解}}
\begin{document}
\courseheader
\name{代卓远2025210205}

\begin{enumerate}
  \setlength{\itemsep}{3\parskip}

  % 补充题
  \item 设$A\in\mathbb{R}^{n\times n}$,$x\in\mathbb{R}^{n}$.证明:$\|Ax\|_{2}\leq\|A\|_{F}\|x\|_{2}$
  \begin{proof}
    设$A=[a_{ij}]_{i,j=1}^{n}$,特征值为$\lambda_i$,\\则$A^TA=\begin{bmatrix}
      \sum_{1}^{n}a_{i1}^2&\sum_{1}^{n}a_{i1}a_{i2}&\cdots&\sum_{1}^{n}a_{i1}a_{in}\\
      \sum_{1}^{n}a_{i2}a_{i1}&\sum_{1}^{n}a_{i2}^2&\cdots&\sum_{1}^{n}a_{i2}a_{in}\\
      \vdots&\vdots&\ddots&\vdots\\
      \sum_{1}^{n}a_{in}a_{1j}&\sum_{1}^{n}a_{in}a_{i2}&\cdots&\sum_{1}^{n}a_{in}^2
      \end{bmatrix}$,特征值为$\lambda_i^2\ge0$\\
    $\mathrm{tr}(A^TA)=\sum_{i,j}a_{ij}=\|A\|_{F}^2\ge \max_{\lambda\in\sigma(A)}\lambda^2=\|A\|_{2}^2$, 所以$\|A\|_{2}\le\|A\|_{F}$\\
    所以$\|Ax\|_{2}\leq\|A\|_{2}\|x\|_{2}\leq\|A\|_{F}\|x\|_{2}$
  \end{proof}

  % 题目8
  \item 已知 $x_0, x_1, x_2, \cdots, x_m \in [a, b]$, 下面的 $(f, g)$ 是否能构成 $C[a, b]$ 上的内积?证明你的结论.
  \begin{enumerate}[label=(\arabic*)]
    \item $(f,g)=\int_{a}^{b}f(x)g(x)\mathrm{d}x,\forall f,g\in C[a,b]$;
    \begin{solution}
      性质如下
      \begin{enumerate}[label=(\roman*)]
        \item 正定性: $(f,f)=\int_{a}^{b}f^2(x)\mathrm{d}x\ge 0$,当且仅当$f(x)=0$时等号成立.
        \item 线性: $(\alpha f+\beta g,h)=\int_{a}^{b}\left[\alpha f(x)+\beta g(x)\right]h(x)\mathrm{d}x=\alpha \int_{a}^{b}f(x)h(x)\mathrm{d}x+\beta \int_{a}^{b}g(x)h(x)\mathrm{d}x=\alpha (f,h)+\beta (g,h)$
        \item 对称性: $(f,g)=\int_{a}^{b}f(x)g(x)\mathrm{d}x=\int_{a}^{b}g(x)f(x)\mathrm{d}x=(g,f)$
      \end{enumerate}
      所以$(f,g)$能构成 $C[a, b]$ 上的内积.
    \end{solution}
    \item $(f,g)=\sum_{i=0}^{m}f(x_{i})g(x_{i}),\forall f,g\in C[a,b].$
    \begin{solution}
      当$x_i=i\pi$,$f(x)=\sin x$,此时$(f,f)=0$,但$f(x)\neq 0$,所以不满足正定性,不能构成内积.
    \end{solution}
  \end{enumerate}

  % 题目11
  \item 求下列矩阵 $A$ 的范数 $\|A\|_1$, $\|A\|_2$, $\|A\|_F$ 和 $\rho(A)$.
  \begin{enumerate}[label=(\arabic*)]
    \item $A=\left[\begin{array}{cc}1&-2\\-3&4\end{array}\right];$
    \begin{solution}
      $A$的特征多项式为
      \begin{align*}
        &det(\lambda I - A) = \begin{vmatrix}\lambda-1&2\\3&\lambda-4\end{vmatrix}= \lambda^2 - 5\lambda - 2=(\lambda-\frac{5+\sqrt{33}}{2})(\lambda-\frac{5-\sqrt{33}}{2})\\
        &\|A\|_1=\max_j\sum_i|a_{ij}|=6,\\
        &\|A\|_2=\sqrt{\rho(A^TA)}=\frac{5+\sqrt{33}}{2},\\
        &\|A\|_F=\sqrt{\sum_{i,j}a_{ij}^2}=\sqrt{30},\\
        &\rho(A)=\max|\lambda_i|=\frac{5+\sqrt{33}}{2}.
      \end{align*}
    \end{solution}
    \item $A=\begin{bmatrix}2&-1&0\\\\-1&2&-1\\\\0&-1&2\end{bmatrix}.$
    \begin{solution}
      $A$的特征多项式为
      \begin{align*}
        &det(\lambda I - A) = \begin{vmatrix}\lambda-2&1&0\\1&\lambda-2&1\\0&1&\lambda-2\end{vmatrix}=(\lambda-2)(\lambda-(2+\sqrt{2}))(\lambda-(2-\sqrt{2}))\\
        &\|A\|_1=\max_j\sum_i|a_{ij}|=4,\\
        &\|A\|_2=\sqrt{\rho(A^TA)}=2+\sqrt{2},\\
        &\|A\|_F=\sqrt{\sum_{i,j}a_{ij}^2}=4,\\
        &\rho(A)=\max|\lambda_i|=2+\sqrt{2}.
      \end{align*}
    \end{solution}
  \end{enumerate}

  % 题目12
  \item 证明
  \begin{enumerate}[label=(\arabic*)]
    \item $\| x \|_\infty \leqslant \| x \|_1 \leqslant n \| x \|_\infty, \forall x \in \mathbb{R}^n,$
    \begin{proof}
      $\max_{1\leq k\leq n}|x_k|\le \sum_{k=1}^n|x_k| \le n\max_{1\leq k\leq n}|x_k|$,即$\| x \|_\infty \leqslant \| x \|_1 \leqslant n \| x \|_\infty$
    \end{proof}
    \item $\| x \|_{\infty} \leqslant \| x \|_{2} \leqslant \sqrt{n} \| x \|_{\infty}, \quad \forall x \in \mathbb{R}^{n};$
    \begin{proof}
      $\max_{1\leq k\leq n}|x_k|\le \sqrt{\sum_{k=1}^n|x_k|^2} \le \sqrt{n}\max_{1\leq k\leq n}|x_k|$,即$\| x \|_{\infty} \leqslant \| x \|_{2} \leqslant \sqrt{n} \| x \|_{\infty}$
    \end{proof}
    \item $\|A\|_2 \leqslant \|A\|_F \leqslant \sqrt{n} \|A\|_2, \quad \forall A \in \mathbb{R}^{n \times n}.$
    \begin{proof}
      由题目2.1可知$tr(A^TA)=\sum_{i,j}a_{ij}^2=\|A\|_F^2\ge \max_{\lambda\in\sigma(A)}\lambda^2=\|A\|_{2}^2$, 所以$\|A\|_{2}\le\|A\|_{F}$\\
      又$tr(A^TA)\le n\max_{\lambda\in\sigma(A)}\lambda^2=n\|A\|_{2}^2$,所以$\|A\|_{F}\le \sqrt{n}\|A\|_{2}$\\
      综上所述$\|A\|_2 \leqslant \|A\|_F \leqslant \sqrt{n} \|A\|_2$
    \end{proof}
  \end{enumerate}

  % 题目15
  \item $A\in \mathbb{R} ^{n\times n}$,设 $A$ 对称正定 ,记
  $$\left\|x\right\|_A\:=\:\sqrt{\left(Ax,x\right)}\:,\quad\forall\:x\:\in\:\mathbb{R}^n\:,$$
  证明$\left\|x\right\|_A$为$\mathbb{R}^n$上的一种向量范数.
  \begin{proof}
    性质如下
    \begin{enumerate}[label=(\roman*)]
      \item 正定性: $\|x\|_A=\sqrt{(Ax,x)}\ge0$,当且仅当$x=0$时等号成立.
      \item 线性: $\|\alpha x\|_A=\sqrt{(A(\alpha x),\alpha x)}=\sqrt{\alpha^2(Ax,x)}=|\alpha|\|x\|_A$
      \item 三角不等式: 
        \begin{align*}
          \|x+y\|_A^2&=(A(x+y),x+y)\\
          &=(Ax,x)+(Ay,y)+(Ax,y)+(Ay,x)\\
          &=(Ax,x)+(Ay,y)+2(Ax,y)
        \end{align*}
        由Cauchy-Schwarz不等式,$(Ax,y)^2\le (Ax,x)(Ay,y)$,即$(Ax,y)\le \|x\|_A\|y\|_A$,所以
        \begin{align*}
          \|x+y\|_A^2&\le \|x\|_A^2+\|y\|_A^2+2\|x\|_A\|y\|_A=(\|x\|_A+\|y\|_A)^2,\\
          \|x+y\|_A&\le \|x\|_A+\|y\|_A
        \end{align*}
        综上所述$\|x\|_A$为$\mathbb{R}^n$上的一种向量范数.
    \end{enumerate}
  \end{proof}
  
  % 题目17
  \item 设$A,\boldsymbol{Q}\in\mathbb{R}^{n\times n},\boldsymbol{Q}$为正交矩阵,证明:
  $$\begin{aligned}&\left\|AQ\right\|_{2}\:=\:\left\|QA\right\|_{2}\:=\:\left\|A\right\|_{2}\:,\\&\left\|AQ\right\|_{F}\:=\:\left\|QA\right\|_{F}\:=\:\left\|A\right\|_{F}.\end{aligned}$$
  \begin{proof}
    对于2-范数
    \begin{align*}
      &\|AQ\|_2=\sqrt{\rho((AQ)^TAQ)}=\sqrt{\rho(AQ(AQ)^T)}=\sqrt{\rho(AQQ^TA^T)}=\sqrt{\rho(AA^T)}=\|A\|_2\\
      &\|QA\|_2=\sqrt{\rho((QA)^TQA)}=\sqrt{\rho(A^TQ^TQA)}=\sqrt{\rho(A^TA)}=\|A\|_2\\
    \end{align*}
    对于Frobenius范数,AQ和QA都是正交变换,则变换后列(行)向量的模长不变,所以
    \begin{align*}
      &\|AQ\|_F=\sqrt{\sum_{i,j}(AQ)_{ij}^2}=\sqrt{\sum_{i,j}a_{ij}^2}=\|A\|_F\\
      &\|QA\|_F=\sqrt{\sum_{i,j}(QA)_{ij}^2}=\sqrt{\sum_{i,j}a_{ij}^2}=\|A\|_F\\
    \end{align*}
  \end{proof}

  % 题目21
  \item 设
  $$A\:=\:\begin{bmatrix}a&1&0\\\\b&2&1\\\\0&1&2\end{bmatrix}\:,$$
  分别求出所有$a,b$的值,使得
  \begin{enumerate}[label=(\arabic*)]
    \item $A$奇异;
    \begin{solution}
      \begin{align*}
        &\det(A)=\begin{vmatrix}a&1&0\\b&2&1\\0&1&2\end{vmatrix}=3a-2b=0\\
        &\Rightarrow b=\frac{3}{2}a
      \end{align*}
    \end{solution}
    \item $A$严格对角占优;
    \begin{solution}
        \begin{align*}
          &|a_{ii}|\geq\sum_{j\neq i}|a_{ij}|\\
          &\Rightarrow 
          \begin{cases}
            |a|\geq1\\
            2\geq|b|+1\\
            2\geq1
          \end{cases}
          \Rightarrow 
          \begin{cases}
            |a|\geq1\\
            |b|\leq1
          \end{cases}
        \end{align*}
    \end{solution}
    \item $A$对称正定.
    \begin{solution}
        对称性: $b=1$ \\
        正定性:主子式均大于0
        \begin{align*}
          \begin{cases}
            b = 1 \\
            a > 0 \\
            \begin{vmatrix}a & 1 \\ 1 & 2\end{vmatrix} = 2a - 1 > 0\\
            \det(A) = 3a - 2 > 0
          \end{cases}
          \Rightarrow
          \begin{cases}
            b = 1 \\
            a > \frac{2}{3}
          \end{cases}
        \end{align*}
    \end{solution}

  \end{enumerate}
\end{enumerate}
\end{document}

%%% Local Variables:
%%% mode: late\rvx
%%% TeX-master: t
%%% End:
