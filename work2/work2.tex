% Homework template for Inference and Information
% UPDATE: September 26, 2017 by Xiangxiang
\documentclass[a4paper]{article}
\usepackage[margin=1in]{geometry}
\usepackage{ctex}
\ctexset{
proofname = \heiti{证明}
}
\usepackage{amsmath, amssymb, amsthm}
% amsmath: equation*, amssymb: mathbb, amsthm: proof
\usepackage{moreenum}
\usepackage{mathtools}
\usepackage{url}
\usepackage{bm}
\usepackage{enumitem}
\usepackage{graphicx}
\usepackage{subcaption}
\usepackage{booktabs} % toprule
\usepackage[mathcal]{eucal}
\usepackage[thehwcnt = 2]{iidef} % 作业编号
\everymath{\displaystyle}


\thecourseinstitute{清华大学机械工程系}
\thecoursename{数值分析A}
\theterm{2025年秋季学期}
\hwname{作业}
\slname{\heiti{解}}
\begin{document}
\courseheader
\name{代卓远2025210205}

\begin{enumerate}
  \setlength{\itemsep}{3\parskip}

  % 补充题
  \item 设$A\in\mathbb{R}^{n\times n}$,$x\in\mathbb{R}^{n}$.证明:$\|Ax\|_{2}\leq\|A\|_{F}\|x\|_{2}$
  \begin{proof}
    设$A=[a_{ij}]_{i,j=1}^{n}$,则
    $A^TA=\begin{bmatrix}
      \sum_{1}^{n}a_{i1}^2&\sum_{1}^{n}a_{i1}a_{i2}&\cdots&\sum_{1}^{n}a_{i1}a_{in}\\
      \sum_{1}^{n}a_{i2}a_{i1}&\sum_{1}^{n}a_{i2}^2&\cdots&\sum_{1}^{n}a_{i2}a_{in}\\
      \vdots&\vdots&\ddots&\vdots\\
      \sum_{1}^{n}a_{in}a_{1j}&\sum_{1}^{n}a_{in}a_{i2}&\cdots&\sum_{1}^{n}a_{in}^2
      \end{bmatrix}$,
    $\mathrm{tr}(A^TA)=\sum_{i,j}$
  \end{proof}

  % 题目8
  \item 已知 $x_0, x_1, x_2, \cdots, x_m \in [a, b]$, 下面的 $(f, g)$ 是否能构成 $C[a, b]$ 上的内积?证明你的结论.
  \begin{enumerate}[label=(\arabic*)]
    \item $(f,g)=\int_{a}^{b}f(x)g(x)\mathrm{d}x,\forall f,g\in C[a,b]$;
    \item $(f,g)=\sum_{i=0}^{m}f(x_{i})g(x_{i}),\forall f,g\in C[a,b].$
    \begin{solution}
      
    \end{solution}
    
  \end{enumerate}

  % 题目11
  \item 求下列矩阵 $A$ 的范数 $\|A\|_1$, $\|A\|_2$, $\|A\|_F$ 和 $\rho(A)$.
  \begin{enumerate}[label=(\arabic*)]
    \item $A=\left[\begin{array}{cc}1&-2\\-3&4\end{array}\right];$
    \begin{solution}
      
    \end{solution}
    \item $A=\begin{bmatrix}2&-1&0\\\\-1&2&-1\\\\0&-1&2\end{bmatrix}.$
    \begin{solution}
      
    \end{solution}
  \end{enumerate}

  % 题目12
  \item 证明
  \begin{enumerate}[label=(\arabic*)]
    \item $\| x \|_\infty \leqslant \| x \|_1 \leqslant n \| x \|_\infty, \forall x \in \mathbb{R}^n,$
    \begin{proof}
      
    \end{proof}
    \item $\| x \|_{\infty} \leqslant \| x \|_{2} \leqslant \sqrt{n} \| x \|_{\infty}, \quad \forall x \in \mathbb{R}^{n};$
    \begin{proof}
      
    \end{proof}
    \item $\|A\|_2 \leqslant \|A\|_F \leqslant \sqrt{n} \|A\|_2, \quad \forall A \in \mathbb{R}^{n \times n}.$
    \begin{proof}
      
    \end{proof}
  \end{enumerate}

  % 题目15
  \item $A\in \mathbb{R} ^{n\times n}$,设 $A$ 对称正定 ,记
  $$\left\|x\right\|_A\:=\:\sqrt{\left(Ax,x\right)}\:,\quad\forall\:x\:\in\:\mathbb{R}^n\:,$$
  证明$\left\|x\right\|_A$为$\mathbb{R}^n$上的一种向量范数.
  \begin{proof}
    
  \end{proof}
  
  % 题目17
  \item 设$A,\boldsymbol{Q}\in\mathbb{R}^{n\times n},\boldsymbol{Q}$为正交矩阵,证明:
  $$\begin{aligned}&\left\|AQ\right\|_{2}\:=\:\left\|QA\right\|_{2}\:=\:\left\|A\right\|_{2}\:,\\&\left\|AQ\right\|_{F}\:=\:\left\|QA\right\|_{F}\:=\:\left\|A\right\|_{F}.\end{aligned}$$
  \begin{proof}
    
  \end{proof}

  % 题目21
  \item 设
  $$A\:=\:\begin{bmatrix}a&1&0\\\\b&2&1\\\\0&1&2\end{bmatrix}\:,$$
  分别求出所有$a,b$的值,使得
  \begin{enumerate}[label=(\arabic*)]
    \item $A$奇异;
    \begin{solution}
      
    \end{solution}
    \item $A$严格对角占优;
    \begin{solution}
      
    \end{solution}
    \item $A$对称正定.
    \begin{solution}
      
    \end{solution}
  \end{enumerate}
\end{enumerate}
\end{document}

%%% Local Variables:
%%% mode: late\rvx
%%% TeX-master: t
%%% End:
