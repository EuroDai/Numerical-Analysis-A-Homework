% Homework template for Inference and Information
% UPDATE: September 26, 2017 by Xiangxiang
\documentclass[a4paper]{article}
\usepackage[margin=1in]{geometry}
\usepackage{ctex}
\ctexset{
proofname = \heiti{证明}
}
\usepackage{amsmath, amssymb, amsthm}
% amsmath: equation*, amssymb: mathbb, amsthm: proof
\usepackage{moreenum}
\usepackage{mathtools}
\usepackage{url}
\usepackage{bm}
\usepackage{enumitem}
\usepackage{graphicx}
\usepackage{subcaption}
\usepackage{booktabs} % toprule
\usepackage[mathcal]{eucal}
\usepackage[thehwcnt = 1]{iidef} % 作业编号
\everymath{\displaystyle}


\thecourseinstitute{清华大学机械工程系}
\thecoursename{数值分析A}
\theterm{2025年秋季学期}
\hwname{作业}
\slname{\heiti{解}}
\begin{document}
\courseheader
\name{代卓远2025210205}

\begin{enumerate}
  \setlength{\itemsep}{3\parskip}

  % 题目2
  \item 已知$4$个四位有效数字的三角函数的值$\sin 1^\circ =0.0175$, $\sin 2^\circ =0.0349$, $\cos 1 ^ \circ = 0.9998$, $\cos 2 ^\circ = 0.9994$.用以下四种方法计算$1 - \cos 2 ^ \circ$的值,比较结果的误差,并说明各有多少位有效数字.
  \begin{enumerate}[label=(\arabic*)]
    % (1)
    \item 直接用已知数字计算;
    \begin{solution}
      直接计算
      \begin{align*}
        &x_A=1 - \cos 2^\circ = 1 - 0.9994 = 10^{-3}\times0.6
        \end{align*}
      误差
        \begin{align*}
        &\left |  x-x_A\right | \le 0.5\times10^{-4}\\
        &n=-3-(-4)=1
      \end{align*}
      有效数字为$1$位.
    \end{solution}
    % (2)
    \item 用公式$1-\cos 2x = 2 \sin^2 x$及已知数据;
    \begin{solution}
      \begin{align*}
        &x_A=1 - \cos 2^\circ = 2\sin^2 1^\circ = 2\times 0.0175^2 =10^{-3}\times0.6125
      \end{align*}
      误差
        \begin{align*}
        &\left |  x-x_A\right | \le 0.5\times10^{-5}\\
        &n=-3-(-5)=2
      \end{align*}
      有效数字为$2$位.
    \end{solution}
    % (3)
    \item 用公式$1-\cos x = \frac{\sin^2 x}{1+\cos x}$及已知数据;
    \begin{solution}
      \begin{align*}
        &x_A=1 - \cos 2^\circ = \frac{\sin^2 2^\circ}{1+\cos 2^\circ} = \frac{0.0349^2}{1+0.9994}\approx 10^{-3}\times0.60918
      \end{align*}
      误差
      \begin{align*}
        &\left |  x-x_A\right | \le 0.5\times10^{-7}\\
        &n=-3-(-7)=4
      \end{align*}
      有效数字为$4$位.
    \end{solution} 
    % (4)
    \item 用$1-\cos x$的Taylor(泰勒)展开式,要计算结果有四位有效数字($1-\cos 2^\circ =6.0917298\cdots \times 10^{-4}$)
    \begin{solution}
      \begin{align*}
        x_A=1 - \cos 2^\circ &= \sum_{i=1}^n (-1)^{i-1}\frac{(2^\circ)^{2i}}{(2i)!}\\
        &\approx \frac{(2\times\frac{\pi}{180})^2}{2!}-\frac{(2\times\frac{\pi}{180})^4}{4!}\\
        &\approx 10^{-3}\times0.60917298
      \end{align*}
      误差
      \begin{align*}
        &\left |  x-x_A\right | \le 0.5\times10^{-7}\\
        &n>-3-(-7)=4
      \end{align*}
      有效数字$>4$位.
    \end{solution}
  \end{enumerate}
  

  % 题目3
  \item 下面是两种利用$9$次Taylor多项式近似计算$e^{-5}$的方法,试分析哪种方法能提供较好的近似值.
  \begin{enumerate}[label=(\arabic*)]
    \item $\mathrm{e}^{-5}\approx\sum_{i=0}^9\left(-1\right)^i\frac{5^i}{i!}$
    \item $\mathrm{e}^{-5}\approx\left(\sum_{i=0}^9\frac{5^i}{i!}\right)^{-1}$
    \begin{solution}
      \begin{align*}
        \begin{tabular}{c|c|c} % <-- Alignments: 1st column left, 2nd middle and 3rd right, with vertical lines in between
          $x$ & $\sum_{i=0}^x\left(-1\right)^i\frac{5^i}{i!}$ & $\left(\sum_{i=0}^x\frac{5^i}{i!}\right)^{-1}$\\
          \hline
          0	&1	&1\\
          1	&-4	&0.166666667\\
          2	&8.5	&0.054054054\\
          3	&-12.33333333	&0.025423729\\
          4	&13.70833333	&0.015296367\\
          5	&-12.33333333	&0.010938924\\
          6	&9.368055556	&0.008840322\\
          7	&-6.132936508	&0.007774898\\
          8	&3.555183532	&0.007230283\\
          9	&-1.827105379	&0.006959453
        \end{tabular}
      \end{align*}
      显然第二种方法更好.
    \end{solution}
    
  \end{enumerate}

  % 题目5
  \item 下列公式要怎样变换才能使数值计算时能避免有效数字的损失?
  \begin{enumerate}[label=(\arabic*)]
    \item $\int_N^{N+1}\frac{1}{1+x^2}\mathrm{d}x=\arctan(N+1)-\arctan N$,$N>>1$;
    \begin{solution}
      \begin{align*}
        &\arctan(N+1)-\arctan N=\arctan\left(\frac{1}{1+(N+1)N}\right)
      \end{align*}
    \end{solution}
    \item $\sqrt{x^2+\frac{1}{x}}-\sqrt{x^2-\frac{1}{x}}$,$|x|>>1$;
    \begin{solution}
      \begin{align*}
        &\sqrt{x^2+\frac{1}{x}}-\sqrt{x^2-\frac{1}{x}}=\frac{2}{x\left(\sqrt{x^2+\frac{1}{x}}+\sqrt{x^2-\frac{1}{x}}\right)}
      \end{align*}
    \end{solution}
    \item $\ln(x+1)-\ln x$,$x>>1$;
    \begin{solution}
      \begin{align*}
        &\ln(x+1)-\ln x=\ln\left(\frac{x+1}{x}\right)=\ln\left(1+\frac{1}{x}\right)
      \end{align*}
    \end{solution}
    \item $\cos^2x-\sin^2x$,$x\approx\frac{\pi}{4}$
    \begin{solution}
      \begin{align*}
        &\cos^2x-\sin^2x=\cos 2x
      \end{align*}
    \end{solution}
  \end{enumerate}

  % 题目7
  \item 已知$I_n=\int_0^1x^n\mathrm{e}^{x-1}\mathrm{d}x,n=0,1,\cdots$,满足$I_0=1-\mathrm{e}^{-1},I_n=1-nI_{n-1},n=1,2,\cdots$.
  \begin{enumerate}[label=(\arabic*)]
    \item 取$I_0$近似值$\tilde{I}_{0}=1-0.3679$,用递推公式$\tilde{I}_{n}=1-n\tilde{I}_{n-1}$计算$I_n$的近似值$\tilde{I}_n,n=1,2,\cdots ,9$(用四位有效数字计算),结果是否准确?
    \begin{solution}
      \begin{align*}
        \begin{tabular}{c|c|c} % <-- Alignments: 1st column left, 2nd middle and 3rd right, with vertical lines in between
          $n$ & $\tilde{I}_n$ & $I_n$\\
          \hline
          0	&0.6321 	&0.632120559\\
          1	&0.3679 	&0.367879441\\
          2	&0.2642 	&0.264241118\\
          3	&0.2074 	&0.207276647\\
          4	&0.1704 	&0.170893412\\
          5	&0.1480 	&0.145532941\\
          6	&0.1120 	&0.126802357\\
          7	&0.2160 	&0.112383504\\
          8	&-0.7280 	&0.100931967\\
          9	&7.552 	&0.091612293
        \end{tabular}
      \end{align*}
      从$n=7$开始,误差过大,不准确.
    \end{solution}
    \item 设$\varepsilon_n=I_n-\tilde{I}_n$,推导$\mid\varepsilon_n\mid$与$\mid\varepsilon_0\mid$的关系.
    \begin{solution}
      \begin{align*}
        &\varepsilon_n=I_n-\tilde{I}_n=1-nI_{n-1}-(1-n\tilde{I}_{n-1})=-n\varepsilon_{n-1}\\
        &\mid\varepsilon_n\mid=n\mid\varepsilon_{n-1}\mid=n(n-1)\mid\varepsilon_{n-2}\mid=\cdots=n!\mid\varepsilon_0\mid
      \end{align*}
    \end{solution}
  \end{enumerate}

\end{enumerate}
\end{document}
\begin{equation}
\end{equation}

%%% Local Variables:
%%% mode: late\rvx
%%% TeX-master: t
%%% End:
