% Homework template for Inference and Information
% UPDATE: September 26, 2017 by Xiangxiang
\documentclass[a4paper]{article}
\usepackage[margin=1in]{geometry}
\usepackage{ctex}
\ctexset{
proofname = \heiti{证明}
}
\usepackage{amsmath, amssymb, amsthm}
\usepackage{mathrsfs}
% amsmath: equation*, amssymb: mathbb, amsthm: proof
\usepackage{extarrows}
\usepackage{moreenum}
\usepackage{mathtools}
\usepackage{url}
\usepackage{bm}
\usepackage{enumitem}
\usepackage{graphicx}
\usepackage{subcaption}
\usepackage{booktabs} % toprule
\usepackage[mathcal]{eucal}
\usepackage[thehwcnt = 11]{iidef} % 作业编号
\everymath{\displaystyle}


\thecourseinstitute{清华大学机械工程系}
\thecoursename{数值分析A}
\theterm{2025年秋季学期}
\hwname{作业}
\slname{\heiti{解}}
\begin{document}
\courseheader
\name{代卓远2025210205}

\begin{enumerate}
  \setlength{\itemsep}{4\parskip}

  % 题目1
  \item 
  \begin{solution}
    
  \end{solution}

  % 题目2
  \item 
  \begin{solution}
    
  \end{solution}

  % 题目5
  \item 求积公式$$\int_0^1f\left(x\right)\mathrm{d}x\:\approx\:C_0f\left(0\right)\:+\:C_1f\left(1\right)\:+\:B_0f^{\prime}\left(0\right)\:,$$已知其余项表达式为$E(f)=kf^{\prime\prime\prime}(\xi),\xi\in(0,1);$试确定求积公式的系数$C_0$, $C_1$和$B_0$并求出$k.$
  \begin{solution}
    由余项表达式可知该求积公式的代数精度为$2$,故对$f\left(x\right)=1,x,x^2$分别求解可得
    \[\begin{cases}
    C_0+C_1=1\\
    C_1+B_0=\frac{1}{2}\\
    C_1+2B_0=\frac{1}{3}
    \end{cases}\Rightarrow\begin{cases}
    C_0=\frac{1}{3}\\
    C_1=\frac{2}{3}\\
    B_0=-\frac{1}6
    \end{cases}\]
    故求积公式为$\int_0^1f\left(x\right)\mathrm{d}x\approx\frac{1}{3}f\left(0\right)+\frac{2}{3}f\left(1\right)-\frac{1}{6}f^{\prime}\left(0\right).$
    \[
    \int_0^1 x^3\mathrm{d}x=\frac14,\quad
    C_0f(0)+C_1f(1)+B_0f'(0)=\frac23\cdot0+\frac13\cdot1+\frac16\cdot0=\frac13
    \]
    误差
    \[
    E=\frac14-\frac13=-\frac1{12}=k\cdot 6
    \Rightarrow\ k=-\frac1{72}.
    \]
  \end{solution}

  % 题目7
  \item 计算定积分$\int_1^2x\ln x$d$x$,若用复合梯形公式来计算,要使误差不超过$10^{-5}$,问区间$[1,2]$要分为多少等份; 若用复合Simpson求积公式来计算, 要达到同样的精度, 区间$[1,2]$应分为多少等份? 若用“端点修正”的复合梯形公式来计算, 要达到同样精度, 区间$[1,2]$应分为多少等份?
  \begin{solution}
    复合梯形公式:
    \[|E_n|\leq\frac{(b-a)^3}{12n^2}\max_{a\leq\xi\leq b}|f^{\prime\prime}(\xi)|=\frac{1}{12n^2}\leq10^{-5}\]
    \[n\geq\sqrt{\frac{10^5}{12}}\approx91.29\]
    故区间$[1,2]$至少要分为$92$等份。
    复合Simpson求积公式:
    \[|E_n|\leq\frac{(b-a)^5}{2880n^4}\max_{a\leq\xi\leq b}|f^{(4)}(\xi)|=\frac{1}{1440n^4}\leq10^{-5}\]
    \[n\geq\sqrt[4]{\frac{10^5}{1440}}\approx2.89\]
    故区间$[1,2]$至少要分为$3$等份。
    端点修正的复合梯形公式:
    \[|E_n|\leq\frac{(b-a)^5}{720^4}\max_{a\leq\xi\leq b}|f^{(4)}(\xi)|=\frac{1}{360n^4}\leq10^{-5}\]
    \[n\geq\sqrt[4]{\frac{10^5}{360}}\approx4.08\]
    故区间$[1,2]$至少要分为$5$等份。
  \end{solution}
\end{enumerate}
\end{document}

%%% Local Variables:
%%% mode: late\rvx
%%% TeX-master: t
%%% End:
