% Homework template for Inference and Information
% UPDATE: September 26, 2017 by Xiangxiang
\documentclass[a4paper]{article}
\usepackage[margin=1in]{geometry}
\usepackage{ctex}
\ctexset{
proofname = \heiti{证明}
}
\usepackage{amsmath, amssymb, amsthm}
\usepackage{mathrsfs}
% amsmath: equation*, amssymb: mathbb, amsthm: proof
\usepackage{extarrows}
\usepackage{moreenum}
\usepackage{mathtools}
\usepackage{url}
\usepackage{bm}
\usepackage{enumitem}
\usepackage{graphicx}
\usepackage{subcaption}
\usepackage{booktabs} % toprule
\usepackage[mathcal]{eucal}
\usepackage[thehwcnt = 10]{iidef} % 作业编号
\everymath{\displaystyle}


\thecourseinstitute{清华大学机械工程系}
\thecoursename{数值分析A}
\theterm{2025年秋季学期}
\hwname{作业}
\slname{\heiti{解}}
\begin{document}
\courseheader
\name{代卓远2025210205}

\begin{enumerate}
  \setlength{\itemsep}{4\parskip}

  % 题目10
  \item 设$f(x)=\sin\frac\pi2x,x\in[-1,1]$,试在[-1,1]上用Legendre多项式作$f$的三次最佳平方逼近多项式.
  \begin{solution}
    用Legendre多项式作三次最佳平方逼近多项式为
    \[s_3(x)=\frac{1}{2}a_0P_0(x)+\frac{3}{2}a_1P_1(x)+\frac{5}{2}a_2P_2(x)+\frac{7}{2}a_3P_3(x),\]
    其中$P_0(x)=1,P_1(x)=x,P_2  (x)=\frac12(3x^2-1),P_3(x)=\frac12(5x^3-3x)$.
    系数$a_i(i=0,1,2,3)$:
    \[a_i=\int_{-1}^1f(x)P_i(x)dx.\]
    计算得:
    \begin{align*}
      a_0&=\int_{-1}^1\sin\frac\pi2xdx=0,\\
      a_1&=\int_{-1}^1x\sin\frac\pi2xdx=\frac{8}{\pi^2},\\
      a_2&=\int_{-1}^1\frac12(3x^2-1)\sin\frac\pi2xdx=0,\\
      a_3&=\int_{-1}^1\frac12(5x^3-3x)\sin\frac\pi2xdx=\frac{48(\pi^2-10)}{\pi^4}.
    \end{align*}
    因此,
    \[s_3(x)=\frac{12}{\pi^2}x+\frac{84(\pi^2-10)}{\pi^4}(5x^3-3x)\]
  \end{solution}

  % 题目11
  \item $f(x) = \ln(1+x)$的Padé逼近$R_{3,2}(x)$.
  \begin{solution}
    $f(x)=\ln(1+x)$的麦克劳林级数展开为
    \[f(x)=x-\frac{x^2}{2}+\frac{x^3}{3}-\frac{x^4}{4}+\frac{x^5}{5}-\cdots.\]
    设$R_{3,2}(x)=\frac{p_0+p_1x+p_2x^2+p_3x^3}{1+q_1x+q_2x^2}$,则有方程组
    \[
      \begin{cases}
        -p_0=0,\\
        1-p_1=0,\\
        q_1-\frac{1}{2}-p_2=0,\\
        q_2-\frac{1}{2}q_1+\frac{1}{3}-p_3=0,\\
        -\frac{1}{2}q_2+\frac{1}{3}q_1-\frac{1}{4}=0,\\
        \frac{1}{3}q_2+\frac{1}{5}=0.
      \end{cases}
    \]
    解得:
    \[p_0=0,p_1=1,p_2=-\frac{13}{20},p_3=-\frac{23}{120},q_1=-\frac{3}{20},q_2=-\frac3{5}.\]
    因此,
    \[R_{3,2}(x)=\frac{x-\frac{13}{20}x^2-\frac{23}{120}}{1-\frac{3}{20}x-\frac3{5}x^2}.\]
  \end{solution}

  % 题目12
  \item 已知数据
    \begin{align*}
      \begin{tabular}{c|cccc}
        \hline
        $x_i$ & -3 & -1 & 1 & 3 \\
        \hline
        $y_i$ & 15 & 5 & 1 & 5 \\
        \hline
      \end{tabular}
    \end{align*}
    试用$y=ax^2+bx+c$来拟合上述数据.
  \begin{solution}
    法方程为
    \[\begin{bmatrix}
      4 & 0 & 20 \\
      0 & 20 & 0 \\
      20 & 0 & 164
    \end{bmatrix}
    \begin{bmatrix}
      c \\ b \\ a
    \end{bmatrix}=
    \begin{bmatrix}
      26 \\ -34 \\ 186
    \end{bmatrix}.\]
    解得:$a=\frac{7}{8},b=-\frac{17}{10},c=\frac{17}8$.
    因此拟合多项式为
    \[y=\frac{7}{8}x^2-\frac{17}{10}x+\frac{17}{8}.\]
  \end{solution}

  % 题目16
  \item 给定数据
    \begin{align*}
      \begin{tabular}{c|cccc}
        \hline
        $x_{i}$ & 0.5 & 1.0 & 1.5 & 2.0 \\
        \hline
        $y_{i}$ & 0.4000 & 0.3333 & 0.2857 & 0.2500 \\
        \hline
      \end{tabular}
    \end{align*}
    试用线性化方法以$y=\frac{a}{x+b}$的形式拟合上述数据.
  \begin{solution}
    对方程线性化,得到
    \[\frac{1}{y}=\frac{b}{a}+\frac{1}{a}x.\]
    设$\frac{1}{y}=Y,\frac{1}{a}=A,\frac{b}{a}=B$,则有$Y=Ax+B $.
    用最小二乘法拟合得到法方程:
    \[\begin{bmatrix}
      4 & 5 \\
      5 & 7.5 \\
    \end{bmatrix}
    \begin{bmatrix}
      B \\ A \\ 
    \end{bmatrix}=
    \begin{bmatrix}
      11.5 \\ 15.25
    \end{bmatrix}.\]
    得$Y=0.7x+2$
    即$y=\frac{1}{0.7x+2}$.
  \end{solution}

  % 题目8
  \item 给定数据
    \begin{align*}
      \begin{tabular}{c|ccccc}
        \hline
        $x_{i}$ & -1 & $-\frac{1}{2}$ & 0 & $\frac{1}{2}$ & 1 \\
        \hline
        $y_{i}$ & -1 & 0 & 1 & 2 & 1 \\
        \hline
      \end{tabular}
    \end{align*}
    试用$y = a \sin \pi x + b \cos \pi x$形式来拟合上述数据.
  \begin{solution}
    由最小二乘法拟合得到法矩阵:
    \[\begin{bmatrix}
        \sum_{i=0}^4\sin^2\pi x_i & \sum_{i=0}^4\sin\pi x_i\cos\pi x_i \\
        \sum_{i=0}^4\sin\pi x_i\cos\pi x_i & \sum_{i=0}^4\cos^2\pi x_i
      \end{bmatrix}=
      \begin{bmatrix}
        2 & 0 \\
        0 & 3
      \end{bmatrix},\]
    则法方程为:
    \[\begin{bmatrix}
        2 & 0 \\
        0 & 3
      \end{bmatrix}
      \begin{bmatrix}
        a \\ b
      \end{bmatrix}=
      \begin{bmatrix}
        \sum_{i=0}^4 y_i\sin\pi x_i \\ \sum_{i=0}^4 y_i\cos\pi x_i
      \end{bmatrix}=
      \begin{bmatrix}
        2 \\ 1
      \end{bmatrix}.\]
      解得$y=\sin \pi x+\frac{1}{3}\cos \pi x$.
  \end{solution}
\end{enumerate}
\end{document}

%%% Local Variables:
%%% mode: late\rvx
%%% TeX-master: t
%%% End:
