% Homework template for Inference and Information
% UPDATE: September 26, 2017 by Xiangxiang
\documentclass[a4paper]{article}
\usepackage[margin=1in]{geometry}
\usepackage{ctex}
\ctexset{
proofname = \heiti{证明}
}
\usepackage{amsmath, amssymb, amsthm}
% amsmath: equation*, amssymb: mathbb, amsthm: proof
\usepackage{moreenum}
\usepackage{mathtools}
\usepackage{url}
\usepackage{bm}
\usepackage{enumitem}
\usepackage{graphicx}
\usepackage{subcaption}
\usepackage{booktabs} % toprule
\usepackage[mathcal]{eucal}
\usepackage[thehwcnt = 3]{iidef} % 作业编号
\everymath{\displaystyle}


\thecourseinstitute{清华大学机械工程系}
\thecoursename{数值分析A}
\theterm{2025年秋季学期}
\hwname{作业}
\slname{\heiti{解}}
\begin{document}
\courseheader
\name{代卓远2025210205}

\begin{enumerate}
  \setlength{\itemsep}{3\parskip}

  % 题目3
  \item 将第 1 题的系数矩阵作 Doolittle 分解$:A=LU.$用直接三角分解方法解第 1 题的方程组 ,并求$\det A.$
  $$\left[\begin{array}{cccc} 6 & 2 & 1 & -1 \\ 2 & 4 & 1 & 0 \\ 1 & 1 & 4 & -1 \\ -1 & 0 & -1 & 3 \end{array}\right]\left[\begin{array}{c} x_{1} \\ x_{2} \\ x_{3} \\ x_{4} \end{array}\right]=\left[\begin{array}{c} 6 \\ 1 \\ 5 \\ -5 \end{array}\right]$$
  \begin{solution}
    \begin{align*}
      &a_{11}=6>0,\quad
      \begin{vmatrix}
        6 & 2 \\ 
        2 & 4
      \end{vmatrix}=20>0,\quad
      \begin{vmatrix}
        6 & 2 & 1 \\ 
        2 & 4 & 1 \\ 
        1 & 1 & 4
      \end{vmatrix}=78>0,\quad
      \det(A)=191>0\\
    \end{align*}
    因此,$A$可LU分解得
    $$A=LU=\begin{bmatrix} 1 & 0 & 0 & 0 \\ 1/3 & 1 & 0 & 0 \\ 1/6 & 1/5 & 1 & 0 \\ -1/6 & 1/10 & -9/37 & 1 \end{bmatrix} \begin{bmatrix} 6 & 2 & 1 & -1 \\ 0 & 10/3 & 2/3 & 1/3 \\ 0 & 0 & 37/10 & -9/10 \\ 0 & 0 & 0 & 191/74 \end{bmatrix}$$
    $Ly=b,Ux=y$得
    $$x=\begin{bmatrix} 151/191 \\ -69/191 \\165/191 \\ -213/191 \end{bmatrix}$$
  \end{solution}

  % 题目8
  \item 用 Cholesky 方法 (平方根法) 解方程组
  $$\begin{bmatrix}
  16 & 4 & 8 \\
  4 & 5 & -4 \\
  8 & -4 & 22
  \end{bmatrix}
  \begin{bmatrix}
  x_1 \\
  x_2 \\
  x_3
  \end{bmatrix}
  =
  \begin{bmatrix}
  -4 \\
  3 \\
  10
  \end{bmatrix}.$$
  \begin{solution}
    $$LU=\begin{bmatrix}1&0&0\\1/4&1&0\\1/2&-3/2&1\end{bmatrix}\begin{bmatrix}16&4&8\\0&4&-6\\0&0&9\end{bmatrix}=
    \begin{bmatrix}
      4 & 0 & 0 \\
      1 & 2 & 0 \\
      2 & -3 & 3
    \end{bmatrix}\begin{bmatrix}
      4 & 0 & 0 \\
      1 & 2 & 0 \\
      2 & -3 & 3
    \end{bmatrix}^T$$
    $Ly=b,Ux=y$得
    $$x=\begin{bmatrix} -9/4 \\ 4 \\2 \end{bmatrix}$$
  \end{solution}

  % 题目15
  \item 已知 $A = \begin{bmatrix} 1 & 1 \\ -5 & 1 \end{bmatrix}$,$B = \begin{bmatrix} 2 & -1 & 0 \\ -1 & 2 & -1 \\ 0 & -1 & 2 \end{bmatrix}$,试求 $\operatorname{cond}(A)_{\infty}$ 和 $\operatorname{cond}(B)_{2}$。
  \begin{solution}
    $$\text{cond}(A)_\infty=\|A\|_\infty\|A^{-1}\|_\infty=6$$
    $$\text{cond}(B)_2=\|B\|_2\|B^{-1}\|_2=3+2\sqrt{2}$$
  \end{solution}


  % 题目18
  \item 设$A\in\mathbb{R}^{n\times n}$, $A$非奇异,且$\|A^{-1}\|\|\delta A\|<1$,试证明$(A+\delta A)^{-1}$存在,且
  $$\frac{\|A^{-1}-(A+\delta A)^{-1}\|}{\|A^{-1}\|}\leqslant\frac{\text{cond}(A)\frac{\|\delta A\|}{\|A\|}}{1-\text{cond}(A)\frac{\|\delta A\|}{\|A\|}}.$$
  \begin{proof}
    $A+\delta A$非奇异,且
    \begin{align*}
      A^{-1}-(A+\delta A)^{-1}=(A^{-1}(A+\delta A)-I)(A+\delta A)^{-1}=A^{-1}\delta A(A+\delta A)^{-1}
    \end{align*}
    因此
    \begin{align*}
      \|A^{-1}-(A+\delta A)^{-1}\|&\leqslant\|A^{-1}\|\|\delta A\|\|(A+\delta A)^{-1}\|\\
      &\leqslant\|A^{-1}\|\|\delta A\|\frac{\|A^{-1}\|}{1-\|A^{-1}\|\|\delta A\|}\\
      &=\frac{\text{cond}(A)\frac{\|\delta A\|}{\|A\|}}{1-\text{cond}(A)\frac{\|\delta A\|}{\|A\|}}\|A^{-1}\|
    \end{align*}

  \end{proof}  

  % 题目19
  \item 设 $A \in \mathbb{R}^{n \times n}, b \in \mathbb{R}^n$, $A$ 非奇异, $x$ 为方程组 $Ax=b$ 的解. 今 $A$ 有误差 $\delta A = \alpha A$, $b$ 有误差 $\delta b = \beta b$, 其中 $\alpha, \beta \in \mathbb{R}$, 满足 $|\alpha| < 1$, $|\beta| < 1$, 且 $\det(A + \delta A) \neq 0$. 向量 $x + \delta x$ 满足
  $$(A + \delta A)(x + \delta x) = b + \delta b,$$
  试证明相对误差估计式
  $$
  \frac{\|\delta x\|_2}{\|x\|_2} \leqslant \frac{|\alpha| + |\beta|}{1 - |\alpha|}.$$
  \begin{proof}
    $$\begin{aligned}\frac{\|\delta x\|}{\|x\|}\leq\frac{\|A^{-1}\|\|A\|}{1-\|A^{-1}\|\|\delta A\|}\left(\frac{\|\delta b\|}{\|b\|}+\frac{\|\delta A\|}{\|A\|}\right)=\frac{\|A^{-1}\|\|A\|}{1-|\alpha| \|A^{-1}\|\|A\|}\left(|\alpha|+|\beta|\right)\end{aligned}$$
    由于$\frac{x}{1-\|\alpha\|x}$单调递增,又$\|A^{-1}\|\|A\|=\text{cond}(A)\geq1$,因此
    $$\frac{\|A^{-1}\|\|A\|}{1-|\alpha| \|A^{-1}\|\|A\|}\leq\frac{1}{1-|\alpha|}$$
    综上所述,得证。
  \end{proof}

\end{enumerate}
\end{document}

%%% Local Variables:
%%% mode: late\rvx
%%% TeX-master: t
%%% End:
