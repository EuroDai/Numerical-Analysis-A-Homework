% Homework template for Inference and Information
% UPDATE: September 26, 2017 by Xiangxiang
\documentclass[a4paper]{article}
\usepackage[margin=1in]{geometry}
\usepackage{ctex}
\ctexset{
proofname = \heiti{证明}
}
\usepackage{amsmath, amssymb, amsthm}
% amsmath: equation*, amssymb: mathbb, amsthm: proof
\usepackage{extarrows}
\usepackage{moreenum}
\usepackage{mathtools}
\usepackage{url}
\usepackage{bm}
\usepackage{enumitem}
\usepackage{graphicx}
\usepackage{subcaption}
\usepackage{booktabs} % toprule
\usepackage[mathcal]{eucal}
\usepackage[thehwcnt = 5]{iidef} % 作业编号
\everymath{\displaystyle}


\thecourseinstitute{清华大学机械工程系}
\thecoursename{数值分析A}
\theterm{2025年秋季学期}
\hwname{作业}
\slname{\heiti{解}}
\begin{document}
\courseheader
\name{代卓远2025210205}

\begin{enumerate}
  \setlength{\itemsep}{4\parskip}

  % 题目5
  \item 设 $f\in C^{1}(\mathbb{R})$,满足 $f(x^{*})=0$,$0<m\leqslant f'(x)\leqslant M$。试证明迭代法$$x_{k+1}=x_{k}-\lambda f(x_{k})$$产生的序列 $\{x_{k}\}$ 对任意的 $x_{0}\in\mathbb{R}$ 及 $\lambda\in(0,\frac{2}{M})$ 均收敛到 $x^{*}$。
  \begin{proof}
    \[
    x_{k+1}-x^{*}=x_{k}-x^{*}-\lambda(f(x_{k})-f(x^{*}))\xlongequal[]{\text{Lagrange}}\left(1-\lambda f'(\xi_{k})\right)(x_{k}-x^{*})
    \]
    其中 $\xi_{k}$ 在 $x_{k}$ 与 $x^{*}$ 之间。因为收敛所以
    \[
    |x_{k+1}-x^{*}|=|1-\lambda f'(\xi_{k})||x_{k}-x^{*}|\le |x_{k}-x^{*}|
    \]
    \[
    -1\le 1-\lambda f'(\xi_{k})\le 1 \Rightarrow 0\le \lambda \le \frac{2}{f'(\xi_{k})}
    \]
    因为$(0,\frac{2}{M})\subseteq [0,\frac{2}{f'(\xi_{k})}]$\\
    所以$\{x_{k}\}$ 对任意的 $x_{0}\in\mathbb{R}$ 及 $\lambda\in(0,\frac{2}{M})$ 均收敛到 $x^{*}$。
  \end{proof}

  % 题目8
  \item 用 Newton 迭代法和割线法求方程 $3x^{2}-e^{x}=0$ 在 $[0,1]$ 上的根
  \begin{solution}
    设$f(x)=3x^{2}-e^{x}$,则$f'(x)=6x-e^{x}$\\
    Newton 迭代法:
    \[
    x_{k+1}=x_{k}-\frac{f(x_{k})}{f'(x_{k})}=x_{k}-\frac{3x_{k}^{2}-e^{x_{k}}}{6x_{k}-e^{x_{k}}}
    \]
    取初值$x_{0}=0.5$,则有
    \begin{align*}
      x_{1}&=1.165\\
      x_{2}&=0.936\\
      x_{3}&=0.910\\
      x_{4}&=0.910
    \end{align*}
    割线法:
    \[
    x_{k+1}=x_{k}-f(x_{k})\frac{x_{k}-x_{k-1}}{f(x_{k})-f(x_{k-1})}
    \]
    取初值$x_{0}=0,x_{1}=1$,则有
    \begin{align*}
      x_{2}&=0.780\\
      x_{3}&=0.903\\
      x_{4}&=0.910\\
      x_{5}&=0.910
    \end{align*}
  \end{solution}

  % 题目13
  \item 
  \begin{enumerate}[label=(\arabic*)]
    \item 为求 $f(x)=0$ 的根, 用迭代函数 $\varphi(x)=x+f(x)$ 的迭代法不一定收敛. 对此用 Steffensen 迭代法, 试写出迭代公式.
    \begin{solution}
      Steffensen 迭代法的迭代公式为
      \[
      x_{k+1}=x_{k}-\frac{[f(x_{k})]^{2}}{f(x_{k}+f(x_{k}))-f(x_{k})}
      \]
    \end{solution}
    \item 设$f$有连续的二阶导数 ,$f(x^*)=0,f^\prime(x^*)\neq0$,研究迭代法$$x_{k+1}\:=\:x_{k}\:-\frac{\left[f(\:x_{k}\:)\:\right]^{2}}{f(\:x_{k}\:+f(\:x_{k}\:)\:)\:-f(\:x_{k}\:)}\:,\quad k\:=\:0\:,1\:,\cdots $$的收敛性和收敛阶。
    \begin{solution}
      \[
      \nu (x)=x-\frac{[f(x)]^{2}}{f(x+f(x))-f(x)}
      \]
      \[
      \nu'(x)=1-\left[ \frac{f(x)}{\frac{f(x+f(x))-f(x)}{x+f(x)-x}}\right]'
      \]
      \[
      \lim_{x\to x^{*}}\nu'(x)=1-\lim_{x\to x^{*}}\left[ \frac{f(x)}{\frac{f(x+f(x))-f(x)}{x+f(x)-x}}\right]'=0
      \]
      因此迭代法收敛且收敛阶至少为2.
    \end{solution}
  \end{enumerate}


  % 题目19
  \item 构造一种不动点迭代法求方程组$$\begin{cases}
    x_{1}-0.7\sin x_{1}-0.2\cos x_{2}=0 \\
    x_{2}-0.7\cos x_{1}+0.2\sin x_{2}=0
  \end{cases}$$在$(0.5,0.5)^T$ 附近的解, 选$x_0=(0.5,0.5)^T$, 迭代至$x^3$或达到$10^{−3}$的精度. 分析方法的收敛性.
  \begin{solution}
    由牛顿迭代法可得不动点迭代格式:
    \[
    \begin{bmatrix}
      x_{1}^{k+1} \\
      x_{2}^{k+1}
    \end{bmatrix}=
    \begin{bmatrix}
      x_{1}^{k} \\
      x_{2}^{k}
    \end{bmatrix}-
    \begin{bmatrix}
      1-0.7\cos x_{1}^{k} & 0.2\sin x_{2}^{k} \\
      0.7\sin x_{1}^{k} & 1-0.2\cos x_{2}^{k}
    \end{bmatrix}^{-1}
    \begin{bmatrix}
      x_{1}^{k}-0.7\sin x_{1}^{k}-0.2\cos x_{2}^{k} \\
      x_{2}^{k}-0.7\cos x_{1}^{k}+0.2\sin x_{2}^{k}
    \end{bmatrix}
    \]
    取初值$x_{0}=(0.5,0.5)^{T}$,则有
    \begin{align*}
      x^{1}&=\begin{bmatrix}
        0.525880696\\
        0.511810035
      \end{bmatrix}\\
      x^{2}&=\begin{bmatrix}
        0.526982887\\
        0.506079826
      \end{bmatrix}\\
      x^{3}&=\begin{bmatrix}
        0.526309687\\
        0.508790639
      \end{bmatrix}
    \end{align*}
    由于在点$(0.5,0.5)^T$处
    \[
    J=\begin{bmatrix}
      0.3948 & 0.3518\\
      ​0.0973 & 0.8252​
    \end{bmatrix}
    \]
    且$\|J\|_{\infty}=0.2916<1$,所以该迭代格式在$(0.5,0.5)^T$附近收敛.
  \end{solution}
\end{enumerate}
\end{document}

%%% Local Variables:
%%% mode: late\rvx
%%% TeX-master: t
%%% End:
