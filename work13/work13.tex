% Homework template for Inference and Information
% UPDATE: September 26, 2017 by Xiangxiang
\documentclass[a4paper]{article}
\usepackage[margin=1in]{geometry}
\usepackage{ctex}
\ctexset{
proofname = \heiti{证明}
}
\usepackage{amsmath, amssymb, amsthm}
\usepackage{mathrsfs}
% amsmath: equation*, amssymb: mathbb, amsthm: proof
\usepackage{extarrows}
\usepackage{moreenum}
\usepackage{mathtools}
\usepackage{url}
\usepackage{bm}
\usepackage{enumitem}
\usepackage{graphicx}
\usepackage{subcaption}
\usepackage{booktabs} % toprule
\usepackage[mathcal]{eucal}
\usepackage[thehwcnt = 13]{iidef} % 作业编号
\everymath{\displaystyle}


\thecourseinstitute{清华大学机械工程系}
\thecoursename{数值分析A}
\theterm{2025年秋季学期}
\hwname{作业}
\slname{\heiti{解}}
\begin{document}
\courseheader
\name{代卓远2025210205}

\begin{enumerate}
  \setlength{\itemsep}{4\parskip}

  % 题目3
  \item 用梯形方法解初值问题$$\left\{
  \begin{array}{l}
  y^{\prime}=\mathrm{e}^{x} \sin (x y), \quad x \in[0,1], \\
  y(0)=1 .
  \end{array}
  \right.$$若迭代初值为 $y_{n+1}^{(0)}=y_{n}+h f\left(x_{n}, y_{n}\right)$, 试选取步长 $h$ 使迭代格式
  $$y_{n+1}^{(s+1)}=y_{n}+\frac{h}{2}\left[f\left(x_{n}, y_{n}\right)+f\left(x_{n+1}, y_{n+1}^{(s)}\right)\right], \quad s=0,1, \cdots$$是收敛的.
  \begin{solution}
对任意 $n\ge0$,设 $y_{n+1}$ 为方程
\[
y_{n+1}=y_n+\frac{h}{2}\Bigl[f(x_n,y_n)+f(x_{n+1},y_{n+1})\Bigr]
\]
的解。与上式相减得
\[
y_{n+1}^{(s+1)}-y_{n+1}=\frac{h}{2}\Bigl[f(x_{n+1},y_{n+1}^{(s)})-f(x_{n+1},y_{n+1})\Bigr].
\]
由微分中值定理,存在 $\xi_{n+1}^{(s)}$ 介于 $y_{n+1}^{(s)}$ 与 $y_{n+1}$ 之间,使
\[
f(x_{n+1},y_{n+1}^{(s)})-f(x_{n+1},y_{n+1})
=f_y(x_{n+1},\xi_{n+1}^{(s)})(y_{n+1}^{(s)}-y_{n+1}),
\]
\[
\bigl|y_{n+1}^{(s+1)}-y_{n+1}\bigr|
=\frac{h}{2}\bigl|f_y(x_{n+1},\xi_{n+1}^{(s)})\bigr|\ \bigl|y_{n+1}^{(s)}-y_{n+1}\bigr|
\le \frac{hL}{2}\bigl|y_{n+1}^{(s)}-y_{n+1}\bigr|.
\]
可得
\[
\bigl|y_{n+1}^{(s+1)}-y_{n+1}\bigr|
\le \left(\frac{hL}{2}\right)^{s+1}\bigl|y_{n+1}^{(0)}-y_{n+1}\bigr|.
\]
当 $0<\frac{hL}{2}<1$ 时,令 $s\to\infty$ 
\[
\bigl|f_y(x,y)\bigr|=\bigl|x e^x\cos(xy)\bigr|\le x e^x\le e.
\]
可取 $L=e$。由上结论,当
\[
\frac{he}{2}<1
\]
即
\[
h<\frac{2}{e}
\]
关于 $s$ 收敛。
\end{solution}




  % 题目5
  \item 试求出单步法$$\begin{cases}y_{n+1}=y_n+hf(x_{n+1},y_n+hf(x_n,y_n)),\\[2ex]y_0=y(x_0)\end{cases}$$的局部截断误差主项及绝对稳定性区间.
  \begin{solution}
考虑初值问题 $y'=f(x,y),\ y(x_0)=\mu$,假定 $y$ 充分光滑,且 $f$ 对 $y$ 的偏导连续有界。

按局部截断误差的定义:
\[
T_{n+1}
= y(x_{n+1})-\Bigl[y(x_n)+h\,f\!\bigl(x_{n+1},\,y(x_n)+h f(x_n,y(x_n))\bigr)\Bigr].
\]
由于 $y'(x)=f(x,y(x))$,故 $f(x_n,y(x_n))=y'(x_n)$,于是
\[
T_{n+1}=y(x_{n+1})-y(x_n)-h\,f\!\bigl(x_{n+1},\,y(x_n)+h y'(x_n)\bigr).
\]
对 $y(x_{n+1})$ 在 $x_n$ 处作 Taylor 展开:
\[
y(x_{n+1})=y(x_n)+h y'(x_n)+\frac{h^2}{2}y''(\xi_n),\qquad \xi_n\in(x_n,x_{n+1}),
\]
代入得
\[
T_{n+1}
=h y'(x_n)+\frac{h^2}{2}y''(\xi_n)-h\,f\!\bigl(x_{n+1},\,y(x_n)+h y'(x_n)\bigr).
\]
再加减 $h f(x_{n+1},y(x_{n+1}))=h y'(x_{n+1})$,得
\[
T_{n+1}
=-\frac{h^2}{2}y''(\tilde\xi_n)
+h\Bigl(f(x_{n+1},y(x_{n+1}))-f(x_{n+1},y(x_n)+h y'(x_n))\Bigr),\quad \tilde\xi_n\in(x_n,x_{n+1}).
\]
存在 $\zeta_n$ 在
$y(x_{n+1})$ 与 $y(x_n)+h y'(x_n)$ 之间,使
\[
f(x_{n+1},y(x_{n+1}))-f(x_{n+1},y(x_n)+h y'(x_n))
=f_y(x_{n+1},\zeta_n)\Bigl(y(x_{n+1})-y(x_n)-h y'(x_n)\Bigr).
\]
又由 Taylor 展开
\[
y(x_{n+1})-y(x_n)-h y'(x_n)=\frac{h^2}{2}y''(\hat\xi_n),\qquad \hat\xi_n\in(x_n,x_{n+1}),
\]
\[
T_{n+1}=-\frac{h^2}{2}y''(\tilde\xi_n)+O(h^3),
\]
局部截断误差主项为
\[
T_{n+1}\sim -\frac{h^2}{2}\,y''(x_n)
\]
取
\[
y'=\lambda y,
\]
\[
y_{n+1}
= y_n+h\lambda\bigl(y_n+h\lambda y_n\bigr)
=\Bigl(1+\lambda h+(\lambda h)^2\Bigr)y_n.
\]
令 $z=\lambda h$,得到稳定函数
\[
E(z)=1+z+z^2.
\]
绝对稳定性条件为
\[
|E(z)|=|1+z+z^2|<1. \tag{*}
\]
\[
1+z+z^2=z^2+z+1>0
\]
因此
\[
z^2+z<0.
\]
即稳定性区间为
\[
z\in(-1,0).
\]
\end{solution}



  % 题目8
  \item 试求出中点公式$$y_{n+1} = y_n + hf\left(x_n + \frac{h}{2}, y_n + \frac{1}{2}hf(x_n, y_n)\right)$$的局部截断误差主项.
  \begin{solution}
假定 $y$ 是初值问题 $y'=f(x,y)$ 的充分光滑精确解,局部截断误差
\[
T_{n+1}
= y(x_{n+1})-\Bigl[y(x_n)+h\,f\!\left(x_n+\frac h2,\;y(x_n)+\frac h2 f(x_n,y(x_n))\right)\Bigr].
\]
Taylor 展开:
\[
y(x_{n+1})=y_n+h y'(x_n)+\frac{h^2}{2}y''(x_n)+\frac{h^3}{6}y^{(3)}(x_n)+O(h^4).
\]
有
\[
y'(x_n)=f_n,\qquad
y''(x_n)=\bigl(f_x+f_y f\bigr)_n,
\]
\[
y^{(3)}(x_n)=\Bigl(f_{xx}+2f_{xy}f+f_{yy}f^2+f_y(f_x+f_y f)\Bigr)_n.
\]
对 $f$ 在点 $(x_n,y_n)$ 作二元 Taylor 展开,令
\[
\Delta x=\frac h2,\qquad \Delta y=\frac h2 f_n,
\]
则
\[
\begin{aligned}
f\!\left(x_n+\frac h2,\;y_n+\frac h2 f_n\right)
&=f_n+f_x\Delta x+f_y\Delta y
+\frac12\Bigl(f_{xx}\Delta x^2+2f_{xy}\Delta x\Delta y+f_{yy}\Delta y^2\Bigr)+O(h^3)\\
&=f_n+\frac h2\bigl(f_x+f_y f\bigr)_n
+\frac{h^2}{8}\bigl(f_{xx}+2f_{xy}f+f_{yy}f^2\bigr)_n+O(h^3).
\end{aligned}
\]
\[
h\,f\!\left(x_n+\frac h2,\;y_n+\frac h2 f_n\right)
= h f_n+\frac{h^2}{2}\bigl(f_x+f_y f\bigr)_n
+\frac{h^3}{8}\bigl(f_{xx}+2f_{xy}f+f_{yy}f^2\bigr)_n+O(h^4).
\]
\[
T_{n+1}
= \frac{h^3}{6}y^{(3)}(x_n)-\frac{h^3}{8}\bigl(f_{xx}+2f_{xy}f+f_{yy}f^2\bigr)_n+O(h^4).
\]
得局部截断误差主项为
\[
T_{n+1}
=\frac{h^3}{24}\Bigl[\bigl(f_{xx}+2f_{xy}f+f_{yy}f^2\bigr)
+4f_y\bigl(f_x+f_y f\bigr)\Bigr]_{(x_n,y(x_n))}+O(h^4).
\]
\end{solution}



  % 题目9
  \item 对于初值问题$$\begin{cases}y'=-y+x+1,\quad x\in[0,1],\\y(0)=1,\end{cases}$$试证明用中点方法、改进的 Euler方法以及Heun方法(见第6题)来解,对任意的步长$h$均有相同的近似值.
  \begin{solution}
记
\[
f(x,y)=-y+x+1,\qquad x_{n+1}=x_n+h,\qquad y_n\approx y(x_n).
\]
1. 中点方法
设
\[
k_1=f(x_n,y_n)=-y_n+x_n+1.
\]
则
\[
k_2=f\!\left(x_n+\frac h2,\;y_n+\frac h2k_1\right)
= -\left(y_n+\frac h2(-y_n+x_n+1)\right)+\left(x_n+\frac h2\right)+1.
\]
\[
k_2=-(1-\tfrac h2)y_n+(1-\tfrac h2)x_n+1=(1-\tfrac h2)(x_n-y_n)+1.
\]
得
\[
y_{n+1}=y_n+h k_2
= y_n+h\Bigl[(1-\tfrac h2)(x_n-y_n)+1\Bigr]. \tag{A}
\]

2. 改进Euler方法

令$
k_2=f(x_{n+1},\,y_n+h k_1)
= -\bigl(y_n+h(-y_n+x_n+1)\bigr)+(x_n+h)+1.
$
化简:
\[
k_2=-(1-h)y_n+(1-h)x_n+1=(1-h)(x_n-y_n)+1.
\]
更新为
\[
\begin{aligned}
y_{n+1}
&=y_n+\frac h2(k_1+k_2) \\
&=y_n+\frac h2\Bigl[(x_n-y_n+1)+\bigl((1-h)(x_n-y_n)+1\bigr)\Bigr]\\
&=y_n+h\Bigl[(1-\tfrac h2)(x_n-y_n)+1\Bigr],
\end{aligned}
\]
3. Heun方法: 
\[
k_1=f(x_n,y_n),\qquad 
k_2=f\!\left(x_n+\frac{2h}{3},\;y_n+\frac{2h}{3}k_1\right),
\]
\[
y_{n+1}=y_n+h\left(\frac14 k_1+\frac34 k_2\right).
\]
结论:三种方法产生同一递推关系
\[
y_{n+1}=y_n+h\Bigl[(1-\tfrac h2)(x_n-y_n)+1\Bigr]
\]
且$y_0=1$,因此对所有 $n$ 都有三种方法给出相同的近似值。
\end{solution}

\end{enumerate}
\end{document}

%%% Local Variables:
%%% mode: late\rvx
%%% TeX-master: t
%%% End:
