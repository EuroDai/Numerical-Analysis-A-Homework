% Homework template for Inference and Information
% UPDATE: September 26, 2017 by Xiangxiang
\documentclass[a4paper]{article}
\usepackage[margin=1in]{geometry}
\usepackage{ctex}
\ctexset{
proofname = \heiti{证明}
}
\usepackage{amsmath, amssymb, amsthm}
\usepackage{mathrsfs}
% amsmath: equation*, amssymb: mathbb, amsthm: proof
\usepackage{extarrows}
\usepackage{moreenum}
\usepackage{mathtools}
\usepackage{url}
\usepackage{bm}
\usepackage{enumitem}
\usepackage{graphicx}
\usepackage{subcaption}
\usepackage{booktabs} % toprule
\usepackage[mathcal]{eucal}
\usepackage[thehwcnt = 12]{iidef} % 作业编号
\everymath{\displaystyle}


\thecourseinstitute{清华大学机械工程系}
\thecoursename{数值分析A}
\theterm{2025年秋季学期}
\hwname{作业}
\slname{\heiti{解}}
\begin{document}
\courseheader
\name{代卓远2025210205}

\begin{enumerate}
  \setlength{\itemsep}{4\parskip}

  % 题目8
  \item 确定求积公式$$\int_{0}^{1}\sqrt{x}f\left(x\right)\mathrm{d}x\approx A_0f\left(x_0\right)+A_1f\left(x_1\right)$$的节点$x_0,x_1$和系数$A_0,A_1$使该求积公式具有三次代数精度.
  \begin{solution}
    设$2$次正交多项式为
    \[ \omega_2\left(x\right)=x^2+ax+b \]
    则有
    \begin{align*}
      &\int_0^1\sqrt{x}\omega_2\left(x\right)\mathrm{d}x=\frac{2}7+\frac{2}5a+\frac{2}3b=0 \\
      &\int_0^1\sqrt{x}x\omega_2\left(x\right)\mathrm{d}x=\frac{2}9+\frac{2}7a+\frac{2}5b=0
    \end{align*}
    解得$a=-\frac{10}{9},b=\frac{5}{21}$,故
    \[ \omega_2\left(x\right)=x^2-\frac{10}{9}x+\frac{5}{21} \]
    因此节点为$\displaystyle x_0=\frac{35-2\sqrt{70}}{63},x_1=\frac{35+2\sqrt{70}}{63}.$
    \[
    \int_0^1 \sqrt{x}\,dx=\frac{2}{3}=A_0+A_1,\qquad
    \int_0^1 x\sqrt{x}\,dx=\frac{2}{5}=A_0x_0+A_1x_1.
    \]
    解得
    \[
    A_0=\frac{1}{3}-\frac{\sqrt{70}}{150},\qquad
    A_1=\frac{1}{3}+\frac{\sqrt{70}}{150}.
    \]
  \end{solution}

  % 题目10
  \item 用Gauss-Chebyshev求积公式证明$$\int_{-1}^{1} \frac{x^{2}}{\sqrt{1-x^{2}}} \mathrm{~d} x=\frac{\pi}{2}$$
  \begin{solution}
    \[x_k=\cos\frac{2k+1}{4}\pi,\quad A_k=\frac{\pi}{2},\quad k=0,1\]
    则有
    \[x_0=\frac{\sqrt{2}}{2},\quad x_1=-\frac{\sqrt{2}}{2},\quad A_0=A_1=\frac{\pi}{2}.\]
    因此
    \[ \int_{-1}^{1} \frac{x^{2}}{\sqrt{1-x^{2}}} \mathrm{~d} x \approx A_0f\left(x_0\right)+A_1f\left(x_1\right)=\frac{\pi}{2}\cdot\frac12+\frac{\pi}{2}\cdot\frac12=\frac{\pi}{2}. \]
  \end{solution}

  % 题目16
  \item 用奇点分离方法计算
  $$\int_0^1x^{-\frac14}\sin x\mathrm{d}x$$
  提示:令$G(x)=\left\{\begin{array}{ll}x^{-\frac{1}{4}}(\sin x-P(x)), & 0<x\leqslant 1 \\ 0, & x=0.\end{array}\right.$,对$\int_{0}^{1}G(x)dx$用$n=2$的复合Simpson求积公式,$P(x)=x-\frac{x^{3}}{6}$
  \begin{solution}
    由Simpson公式
    \begin{align*}
      S_{n}(f) &= \frac{h}{6} \sum_{k=1}^{n} \left[f(x_{k-1}) + 4f(x_{k-1/2}) + f(x_{k})\right] \\
      &= \frac{h}{6} \left[f(a) + 4 \sum_{k=1}^{n} f(x_{k-1/2}) + 2 \sum_{k=1}^{n-1} f(x_{k}) + f(b)\right]
    \end{align*}
    其中$h=\frac{b-a}{n}=\frac{1}{2}$,$x_0=0,x_1=0.5,x_2=1$,则
    \begin{align*}
      S_2(G)&=\frac{1/2}{6}\left[G(0)+4G(0.25)+2G(0.5)+4G(0.75)+G(1)\right]\\
      &=\frac{1}{12}\left[0+4\cdot0.00001149+2\cdot0.0003079+4\cdot0.002997+0.0081377\right]\\
      &\approx0.0017
    \end{align*}
    \[
      \int_0^1 P(x)x^{-\frac{1}{4}}\mathrm{d}x=\int_0^1 \left(x^{\frac{3}{4}}-\frac{x^{\frac{11}{4}}}{6}\right)\mathrm{d}x=\frac{4}{7}-\frac{2}{45}\approx 0.5270.
    \]
    则有
    \[
      \int_0^1 x^{-\frac{1}{4}}\sin x\mathrm{d}x=\int_0^1 G(x)\mathrm{d}x+\int_0^1 P(x)x^{-\frac{1}{4}}\mathrm{d}x\approx0.0017+0.5270=0.5287.
    \]

  \end{solution}

  % 题目18
  \item 试确定数值微分公式的误差项
  \begin{enumerate}[label=(\arabic*)]
    \item $f^{\prime}(x_{0})\approx\frac{1}{4h}[f(x_{0}+3h)-f(x_{0}-h)]$
    \begin{solution}
      由Taylor展开式
      \begin{align*}
        f(x_0+3h)&=f(x_0)+3hf'(x_0)+\frac{9h^2}{2}f''(\xi_1)\\
        f(x_0-h)&=f(x_0)-hf'(x_0)+\frac{h^2}{2}f''(\xi_2)
      \end{align*}
      故误差项为
      $$f^{\prime}(x_{0})-\frac{1}{4h}[f(x_{0}+3h)-f(x_{0}-h)]=-\frac{9h}{8}f''(\xi_1)+\frac{h}{8}f''(\xi_2)$$
      由于
      \[ -h\max f''\le-\frac{9h}{8}f''(\xi_1)+\frac{h}{8}f''(\xi_2)\le -h\min f'' \]
      故误差项为$-hf''(\xi)$.
    \end{solution}

    \item $f'(x_0)\approx\frac{1}{2h}[4f(x_0+h)-3f(x_0)-f(x_0+2h)]$
    \begin{solution}
      由三阶Taylor展开式
      \begin{align*}
        f(x_0+h)&=f(x_0)+hf'(x_0)+\frac{h^2}{2}f''(x_0)+\frac{h^3}{6}f'''(\xi_1)\\
        f(x_0+2h)&=f(x_0)+2hf'(x_0)+2h^2f''(x_0)+\frac{4h^3}{3}f'''(\xi_2)
      \end{align*}
      故误差项为
      \[ f'(x_0)-\frac{1}{2h}[4f(x_0+h)-3f(x_0)-f(x_0+2h)]= \frac{h^2}{3}f'''(\xi) \]
    \end{solution}
  \end{enumerate}
\end{enumerate}
\end{document}

%%% Local Variables:
%%% mode: late\rvx
%%% TeX-master: t
%%% End:
