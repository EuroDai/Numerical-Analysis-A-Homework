% Homework template for Inference and Information
% UPDATE: September 26, 2017 by Xiangxiang
\documentclass[a4paper]{article}
\usepackage[margin=1in]{geometry}
\usepackage{ctex}
\ctexset{
proofname = \heiti{证明}
}
\usepackage{amsmath, amssymb, amsthm}
% amsmath: equation*, amssymb: mathbb, amsthm: proof
\usepackage{extarrows}
\usepackage{moreenum}
\usepackage{mathtools}
\usepackage{url}
\usepackage{bm}
\usepackage{enumitem}
\usepackage{graphicx}
\usepackage{subcaption}
\usepackage{booktabs} % toprule
\usepackage[mathcal]{eucal}
\usepackage[thehwcnt = 6]{iidef} % 作业编号
\everymath{\displaystyle}


\thecourseinstitute{清华大学机械工程系}
\thecoursename{数值分析A}
\theterm{2025年秋季学期}
\hwname{作业}
\slname{\heiti{解}}
\begin{document}
\courseheader
\name{代卓远2025210205}

\begin{enumerate}
  \setlength{\itemsep}{4\parskip}

  % 题目20(2)
  \item 用Newton迭代法和逆Broyden迭代法解$$\begin{cases}x^2+y^2=4\:,\\x^2-y^2=1\:,\end{cases}x^0=(1.6,1.2)^\mathrm{T}$$
  \begin{solution}
    1. Newton迭代法:
    设
    \[
    F(x)=
    \begin{bmatrix}
      x^2+y^2-4\\
      x^2-y^2-1
    \end{bmatrix},\quad F'(x)=
    \begin{bmatrix}
      2x&2y\\
      2x&-2y
    \end{bmatrix}
    \]
    则
    \[
    \begin{bmatrix}
      x^{k+1}\\
      y^{k+1}
    \end{bmatrix}=
    \begin{bmatrix}
      x^k\\
      y^k
    \end{bmatrix}-F(x^k)^{-1}F(x^k)
    \]
    得
    \begin{align*}
    \begin{bmatrix}x^1\\y^1\end{bmatrix}&=\begin{bmatrix}1.5811\\1.2247\end{bmatrix}
    \end{align*}
    2. 逆Broyden迭代法:
    \[
    F(x^0)=
    \begin{bmatrix}
      0\\
      0.12
    \end{bmatrix}
    \]
    \[
    B_0=F(x^0)^{-1}=
    \begin{bmatrix}
      0.15625&0.15625\\
      0.20833&-0.20833
    \end{bmatrix}
    \]
    \[
    x^{k+1}=x^k-B_kF(x^k)
    \]
    \[
    B_{k+1}=B_k+\frac{(p^k-B_k q^k)(p^k)^\mathrm{T}B_k}{(p^k)^\mathrm{T}B_kq^k}
    \]
    其中$p^k=x^{k+1}-x^k,q^k=F(x^{k+1})-F(x^k)$.
    得
    \begin{align*}
    x^1=
    \begin{bmatrix}
      1.5812\\
      1.2250
    \end{bmatrix},\quad x^2=
    \begin{bmatrix}
      1.5811\\
      1.2247
    \end{bmatrix}
    \end{align*}
  \end{solution}

  % 题目2
  \item 利用Gershgorin定理估计$\text{cond}(A)_2$的上界$$A = \begin{bmatrix} 5.2 & 0.6 & 2.2 \\ 0.6 & 0.4 & 0.5 \\ 2.2 & 0.5 & 4.7 \end{bmatrix}$$
  \begin{solution}
    由Gershgorin定理,矩阵$A$的特征值落在以下圆盘中:
    \begin{align*}
      D_1:|z-5.2|\leq 2.8,\quad D_2:|z-0.4|\leq 1.1,\quad D_3:|z-4.7|\leq 2.7\\
      \Rightarrow \sigma(A)\subseteq [-0.7,1.5] \cup [2,8]
    \end{align*}
    因此,$\text{cond}(A)_2=\frac{|\lambda_{\max}|}{|\lambda_{\min}|}\leq \frac{8}{0.7}\approx 11.43$.
  \end{solution}


  % 题目6
  \item 用逆幂迭代法求矩阵$A=\begin{bmatrix}6&2&1\\2&3&1\\1&1&1\end{bmatrix}$最接近2的特征值及对应的特征向量,结果准确到$10^{-3}.$
  \begin{solution}
    使用原点位移的逆幂迭代法,取$\mu=2$,则
    \[B=A-2I=\begin{bmatrix}4&2&1\\2&1&1\\1&1&-1\end{bmatrix}\]
    \[B^{-1}=\begin{bmatrix}2&-3&-1\\-3&5&2\\-1&2&0\end{bmatrix}\]
    取初始向量$v^{(0)}=\begin{bmatrix}1\\1\\1\end{bmatrix}$,由逆幂迭代法
    $$z=A^{-1}v^{(k)},\quad v^{(k+1)}=\frac z{\|z\|},\quad k=0,1,\cdots.$$
    得
    \begin{align*}
      v^{(1)}=[-0.436, 0.873, 0.218]^T,\quad \lambda^{(1)}-2 &= 0.143,\quad \lambda^{(1)} = 2.143\\
      v^{(2)}=[-0.496, 0.818, 0.292]^T,\quad \lambda^{(2)}-2 &= 0.133,\quad \lambda^{(2)} = 2.133\\
      v^{(3)}=[-0.497, 0.820, 0.284]^T,\quad \lambda^{(3)}-2 &= 0.133,\quad \lambda^{(3)} = 2.133
    \end{align*}
  \end{solution}

  % 题目9
  \item 设$A$有实特征值$\lambda _1> \lambda _2\geq \lambda _3\geq \cdots \geq \lambda _{n- 1}\geq \lambda _n$,求$\lambda_1$的原点位移的幂迭代法引人参数$\alpha$,对 $B=A-\alpha I$作幂迭代.试证明这样的迭代当取$\displaystyle \alpha=\frac{\lambda_2+\lambda_n}2$时收敛最快
  \begin{solution}
    $B$的特征值为$\lambda_1-\alpha,\lambda_2-\alpha,\cdots,\lambda_n-\alpha$, 要求$\lambda_1-\alpha$是主特征值. 因此要使$f(\alpha)=\max\left\{\frac{|\lambda_2-\alpha|}{\mid\lambda_1-\alpha\mid},\frac{|\lambda_n-\alpha\mid}{\mid\lambda_1-\alpha\mid}\right\}$最小, 迭代速度最快, 得当
    $$\lambda_n-\alpha=-\lambda_2+\alpha\Rightarrow \alpha=\frac{\lambda_2+\lambda_n}2$$时收敛最快.
  \end{solution}
\end{enumerate}
\end{document}

%%% Local Variables:
%%% mode: late\rvx
%%% TeX-master: t
%%% End:
