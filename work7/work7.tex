% Homework template for Inference and Information
% UPDATE: September 26, 2017 by Xiangxiang
\documentclass[a4paper]{article}
\usepackage[margin=1in]{geometry}
\usepackage{ctex}
\ctexset{
proofname = \heiti{证明}
}
\usepackage{amsmath, amssymb, amsthm}
% amsmath: equation*, amssymb: mathbb, amsthm: proof
\usepackage{extarrows}
\usepackage{moreenum}
\usepackage{mathtools}
\usepackage{url}
\usepackage{bm}
\usepackage{enumitem}
\usepackage{graphicx}
\usepackage{subcaption}
\usepackage{booktabs} % toprule
\usepackage[mathcal]{eucal}
\usepackage[thehwcnt = 7]{iidef} % 作业编号
\everymath{\displaystyle}


\thecourseinstitute{清华大学机械工程系}
\thecoursename{数值分析A}
\theterm{2025年秋季学期}
\hwname{作业}
\slname{\heiti{解}}
\begin{document}
\courseheader
\name{代卓远2025210205}

\begin{enumerate}
  \setlength{\itemsep}{4\parskip}

  % 题目4
  \item 作$A=\begin{bmatrix}1&1&1\\2&-1&-1\\2&-4&5\end{bmatrix}$的QR分解
  \begin{solution}
    用构建Householder矩阵的方法:
    设$A$的列向量为$\bm{a}_1,\bm{a}_2,\bm{a}_3$,则
    \[
      \bm{a}_1=\begin{bmatrix}1\\2\\2\end{bmatrix},\quad \|\bm{a}_1\|_2=\sqrt{9}=3
    \]
    取$\bm{u}_1=\bm{a}_1-3\bm{e}_1=\begin{bmatrix}-2\\2\\2\end{bmatrix}$,则
    \[
      P_1=I-\frac{2}{\bm{u}_1^T\bm{u}_1}\bm{u}_1\bm{u}_1^T=\begin{bmatrix}
        -\frac{1}{3} & \frac{2}{3} & \frac{2}{3} \\
        \frac{2}{3} & \frac{1}{3} & -\frac{2}{3} \\
        \frac{2}{3} & -\frac{2}{3} & \frac{1}{3}
      \end{bmatrix}
    \]
    计算$P_1A$:
    \[
      P_1A=\begin{bmatrix}
        3 & -3 & 3 \\
        0 & 3 & -3 \\
        0 & 0 & 3
      \end{bmatrix}
    \]
    此时已是上三角矩阵,故$R=P_1A$, $Q=P_1^T=P_1$.    
  \end{solution}

  % 题目11
  \item 用正交相似变换化下列矩阵为上Hessenberg矩阵:
  \begin{enumerate}[label=(\arabic*)]
    \item $\begin{bmatrix} 2 & -1 & 3 \\ 2 & 0 & 1 \\ -2 & 1 & 4 \end{bmatrix}$;
    \begin{solution}
      构造Householder矩阵:
      设$A$的列向量为$\bm{a}_1,\bm{a}_2,\bm{a}_3$,则
      \[
        \bm{a}_1^{(1)}=\begin{bmatrix}a_{21}\\a_{31}\end{bmatrix},=\begin{bmatrix}2\\-2\end{bmatrix},\quad \|\bm{a}_1\|_2=2\sqrt{2}
      \]
      取$\bm{u}_1=\bm{a}_1^{(1)}-2\sqrt{2}\bm{e}_1=\begin{bmatrix}2-2\sqrt{2}\\-2\end{bmatrix}$,则
      \[
        \overline{P}_1 =I-\frac{2}{\bm{u}_1^T\bm{u}_1}\bm{u}_1\bm{u}_1^T=\begin{bmatrix}
          \frac{\sqrt{2}}{2} & -\frac{\sqrt{2}}{2} \\
          -\frac{\sqrt{2}}{2} & -\frac{\sqrt{2}}{2} \\
        \end{bmatrix}
      \]
      则$P_1=\begin{bmatrix}
        1 & 0 \\
        0 & \overline{P}_1 \\
      \end{bmatrix}$
      \[
      A^{(2)}=P_1AP_1=\begin{bmatrix}
        2 & -2\sqrt{2} & -\sqrt{2} \\
        -2\sqrt{2} & 1 & 2 \\
        0 & 2 & 3 \\
      \end{bmatrix}
      \]
    \end{solution}
    \item $\begin{bmatrix} 4 & 1 & -2 & 2 \\ 1 & 2 & 0 & 1 \\ -2 & 0 & 3 & -2 \\ 2 & 1 & -2 & 1 \end{bmatrix}$.
    \begin{solution}
      构造Householder矩阵:
      设$A$的列向量为$\bm{a}_1,\bm{a}_2,\bm{a}_3,\bm{a}_4$,则
      \[
        \bm{a}_1^{(1)}=\begin{bmatrix}a_{21}\\a_{31}\\a_{41}\end{bmatrix},=\begin{bmatrix}1\\-2\\2\end{bmatrix},\quad \|\bm{a}_1\|_2=3
      \]
      取$\bm{u}_1=\bm{a}_1^{(1)}-3\bm{e}_1=\begin{bmatrix}-2\\-2\\2\end{bmatrix}$,则
      \[
        \overline{P}_1 =I-\frac{2}{\bm{u}_1^T\bm{u}_1}\bm{u}_1\bm{u}_1^T=\begin{bmatrix}
        \frac{1}{3} & -\frac{2}{3} & \frac{2}{3} \\
        -\frac{2}{3} & \frac{1}{3} & \frac{2}{3} \\
        \frac{2}{3} & \frac{2}{3} & \frac{1}{3}
      \end{bmatrix}
      \]
      则$P_1=\begin{bmatrix}
        1 & 0 \\
        0 & \overline{P}_1 \\
      \end{bmatrix}$
      \[
      A^{(2)}=P_1AP_1=\begin{bmatrix}
        4 & 3 & 0 & 0 \\
        3 & \frac{38}{9} & -\frac{4}{9} & -\frac{5}{9} \\
        0 & -\frac{4}{9} & -\frac{1}{9} & -\frac{8}{9} \\
        0 & -\frac{5}{9} & -\frac{8}{9} & \frac{17}{9} \\
      \end{bmatrix}
      \]
      \[
      a_2^{(2)}=\begin{bmatrix}a_{32}^{(2)}\\a_{42}^{(2)}\end{bmatrix}=\begin{bmatrix}-\frac{4}{9}\\-\frac{5}{9}\end{bmatrix},\quad \|a_2^{(2)}\|_2=\frac{\sqrt{41}}{9}
      \]
      取$\bm{u}_2=\bm{a}_2^{(2)}-\frac{13}{9}\bm{e}_1=\begin{bmatrix}-\frac{4-\sqrt{41}}{9}\\-\frac{5}{9}\end{bmatrix}$,则
      \[
        \overline{P}_2
        = I_2-\frac{2}{\bm{u}_2^T\bm{u}_2}\bm{u}_2\bm{u}_2^T
        = \frac1{\sqrt{41}}
        \begin{bmatrix}
        -4 & -5\\
        -5 & 4
        \end{bmatrix}.
      \]
      则$P_2=\begin{bmatrix}
        I_2 & 0 \\
        0 & \overline{P}_2 \\
      \end{bmatrix}$
      \[
      A^{(3)}=P_2A^{(2)}P_2=\begin{bmatrix}
        4 & 3 & 0 & 0\\
        3 & \frac{38}{9} & \frac{\sqrt{41}}{9} & 0\\
        0 & \frac{\sqrt{41}}{9} & \frac{89}{369} & -\dfrac{48}{41}\\
        0 & 0 & -\frac{48}{41} & \frac{63}{41}
      \end{bmatrix}.
      \]
      $A^{(3)}$即为所求的上Hessenberg矩阵.
    \end{solution}
  \end{enumerate}



  % 题目13(1)
  \item 设$A=\begin{bmatrix}2&\varepsilon\\\varepsilon&1\end{bmatrix}$, 试计算一步QR迭代. 用基本QR算法.
  \begin{solution}
  对 $A_1$ 做 QR 分解

  取第一列
  \[
    a_1 = \begin{bmatrix}2 \\ \varepsilon\end{bmatrix}, 
    \qquad \|a_1\| = \sqrt{4+\varepsilon^2} =: s.
  \]
  定义
  \[
    q_1 = \frac{1}{s}\begin{bmatrix}2 \\ \varepsilon\end{bmatrix}.
  \]
  取与 $q_1$ 正交的单位向量
  \[
    q_2 = \frac{1}{s}\begin{bmatrix}-\varepsilon \\ 2\end{bmatrix},
  \]
  \[
    Q_1 = Q = \frac{1}{s}
    \begin{bmatrix}
      2 & -\varepsilon \\
      \varepsilon & 2
    \end{bmatrix}.
  \]

  由 $R_1 = Q_1^{T}A_1$ 得
  \[
    Q^{T} = \frac{1}{s}
    \begin{bmatrix}
      2 & \varepsilon \\
      -\varepsilon & 2
    \end{bmatrix},
  \]
  因此
  \[
    R_1 = R = Q^{T}A
    =
    \frac{1}{s}
    \begin{bmatrix}
      2 & \varepsilon \\
      -\varepsilon & 2
    \end{bmatrix}
    \begin{bmatrix}
      2 & \varepsilon \\
      \varepsilon & 1
    \end{bmatrix}
    =
    \begin{bmatrix}
      s & \dfrac{3\varepsilon}{s} \\
      0 & \dfrac{2-\varepsilon^2}{s}
    \end{bmatrix},
  \]
  其中 $s=\sqrt{4+\varepsilon^2}$。
  \[
    A_2 = RQ
    =
    \begin{bmatrix}
      s & \dfrac{3\varepsilon}{s} \\
      0 & \dfrac{2-\varepsilon^2}{s}
    \end{bmatrix}
    \cdot
    \frac{1}{s}
    \begin{bmatrix}
      2 & -\varepsilon \\
      \varepsilon & 2
    \end{bmatrix}
    =
    \begin{bmatrix}
      \dfrac{5\varepsilon^2 + 8}{\varepsilon^2 + 4} &
      \dfrac{\varepsilon(2 - \varepsilon^2)}{\varepsilon^2 + 4} \\[0.6em]
      \dfrac{\varepsilon(2 - \varepsilon^2)}{\varepsilon^2 + 4} &
      \dfrac{2(2 - \varepsilon^2)}{\varepsilon^2 + 4}
    \end{bmatrix}.
  \]
\end{solution}


  % 题目2
  \item 2. 设$f(x)=3x\mathrm{e}^x-2\mathrm{e}^x$,取 $x_0=1.0,x_1=1.05,x_2=1.07$,构造二次 Lagrange 插值多项式$L_2$, 并计算$f(1.03)$的近似值. 给出实际计算误差及估计误差界.
  \begin{solution}
    插值节点为
    \begin{align*}
      &f(x_0)=3\cdot1\cdot \mathrm{e}^1-2\mathrm{e}^1\approx2.71828,\quad
      f(x_1)=3\cdot1.05\cdot \mathrm{e}^{1.05}-2\mathrm{e}^{1.05}=3.28630,\\
      &f(x_2)=3\cdot1.07\cdot \mathrm{e}^{1.07}-2\mathrm{e}^{1.07}=3.52761.
    \end{align*}
    则二次Lagrange插值多项式为
    \begin{align*}
      L_2(x)=&2.71828\cdot\frac{(x-1.05)(x-1.07)}{(1.0-1.05)(1.0-1.07)}
      +3.28630\cdot\frac{(x-1.0)(x-1.07)}{(1.05-1.0)(1.05-1.07)}\\
      &+3.52761\cdot\frac{(x-1.0)(x-1.05)}{(1.07-1.0)(1.07-1.05)}
    \end{align*}
    \[
      \Rightarrow L_2(1.03)\approx3.0525.
    \]
    计算误差
    \[
      R_2(1.03)=f(1.03)-3.0525\approx0.000114
    \]
    估计误差界:
    \begin{align*}
    h = \max\{|1.05-1.0|,|1.07-1.05|\}=0.05\\
    \|f-L_n\|_{\infty}\leqslant\frac{h^{n+1}}{4(n+1)}\|f^{(n+1)}\|_{\infty}=\frac{0.05^3}{12}\max_{x\in[1,1.07]}|f^{(3)}(x)|\approx0.0007
    \end{align*}
  \end{solution}

  % 题目6
  \item 给定数据$$\begin{array}{|c|c|c|c|c|} \hline x & 1 & 1.5 & 0 & 2 \\ \hline f(x) & 3 & 3.25 & 3 & 5/3 \\ \hline \end{array}$$试构造出$f$的均差表和三次Newton插值多项式,并写出均差型余项.
  \begin{solution}
    \[
      \begin{array}{|c|c|c|c|c|} \hline
        x & 1 & 1.5 & 0 & 2 \\ \hline
        f[x_i] & 3 & 3.25 & 3 & 5/3 \\ \hline
        f[x_i,x_{i+1}] & 0.5 & -0.5 & -\frac{2}{3} \\ \hline
        f[x_i,x_{i+1},x_{i+2}] & 1 & -\frac{2}{3} \\ \hline
        f[x_i,x_{i+1},x_{i+2},x_{i+3}] & -\frac{5}{3} \\ \hline
      \end{array}
    \]
    则三次Newton插值多项式为
    \[
      N_3(x)=3+0.5(x-1)+1(x-1)(x-1.5)-\frac{5}{3}(x-1)(x-1.5)x
    \]
    均差型余项为
    \[
      R_3(x)=f[x_0,x_1,x_2,x_3,x]\cdot(x-1)(x-1.5)x(x-2)
    \]
  \end{solution}
\end{enumerate}
\end{document}

%%% Local Variables:
%%% mode: late\rvx
%%% TeX-master: t
%%% End:
